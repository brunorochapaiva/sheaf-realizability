\documentclass[11pt]{article}

\newif\ifnotes
\notestrue % comment out to hide notes

\DeclareUnicodeCharacter{03A3}{\(\Sigma\)}
\DeclareUnicodeCharacter{03A0}{\(\Pi\)}
\DeclareFontFamily{U}{min}{}
\DeclareFontShape{U}{min}{m}{n}{<-> udmj30}{}
\newcommand\yo{\!\text{\usefont{U}{min}{m}{n}\symbol{'207}}\!}

\usepackage{geometry}
\geometry{
  a4paper,
  left=25mm,
  right=25mm,
  top=25mm,
}

% Packages for using unicode
\usepackage[utf8]{inputenc}
\usepackage[english]{babel}

\usepackage[T1]{fontenc}
%\usepackage[scaled=0.75]{DejaVuSansMono} % Good monospace font for code

\usepackage[
  backend=biber,
  mincrossrefs=999,
  style=numeric,
  doi=true,
  isbn=false,
  url=false,
  eprint=false,
]{biblatex}
\addbibresource{references.bib}
\AtEveryBibitem{\clearname{editor}} % Always remove editor information

\usepackage[]{hyperref}
\hypersetup{
    colorlinks=false,
}

\usepackage{amsmath}
\usepackage{amssymb}
\usepackage{amsfonts}
\usepackage{amsthm}
\usepackage{newtxmath}
\usepackage{url}
\usepackage{tikz-cd}
\usepackage{quiver}
\usepackage{scalefnt}
\usepackage{microtype}
\usepackage{subfiles}
\usepackage{mathpartir}
\usepackage[bb=dsserif]{mathalpha}
\usepackage{bm}
\usepackage{mathtools}
\ifnotes%
\usepackage{todonotes}
\else
\usepackage[disable]{todonotes}
\fi
\usepackage{cleveref}
\usepackage{xspace}
\usepackage{multirow}
\usepackage{ulem}
\usepackage{contour}


\newtheorem{thrm}{Theorem}[section]
\newtheorem{lemm}[thrm]{Lemma}
\newtheorem{prop}[thrm]{Proposition}
\newtheorem{defn}[thrm]{Definition}
\newtheorem{remk}[thrm]{Remark}
\newtheorem{exam}[thrm]{Example}
\newtheorem{cons}[thrm]{Construction}
\newtheorem{coro}[thrm]{Corollary}
\newtheorem{nota}[thrm]{Notation}

%% Macros

%% Macro to superimpose two symbols
\makeatletter
\newcommand{\superimpose}[3][\mathord]{#1{\mathpalette\superimpose@{{#2}{#3}}}}
\newcommand{\superimpose@}[2]{\superimpose@@{#1}#2}
\newcommand{\superimpose@@}[3]{%
  \ooalign{%
    \hfil$\m@th#1#2$\hfil\cr
    \hfil$\m@th#1#3$\hfil\cr
  }%
}
\makeatother

%%%%%%%%%%%%%%%%%%%%%%%%%%%%%%%%%%%%%%%%%%%%%%%%%%%%%%%%%%%%%%%%%%%%%%%%%%%%%%%%
%% GENERAL MATHEMATICAL NOTATION %%%%%%%%%%%%%%%%%%%%%%%%%%%%%%%%%%%%%%%%%%%%%%%
%%%%%%%%%%%%%%%%%%%%%%%%%%%%%%%%%%%%%%%%%%%%%%%%%%%%%%%%%%%%%%%%%%%%%%%%%%%%%%%%

%% Used in the vernacular to mark a term being defined.
%% Currently, does nothing else besides making the term italic.
%% But it’s still nice to provide a layer of abstraction for this purpose.
%% It may be used, for example, to add things to the index.
\newcommand{\definiendum}[1]{\textbf{#1}}

\newcommand{\IsDefined}{\mathbin{\vcentcolon\equiv}}

%% General one letter notations
\newcommand{\A}{\ensuremath{\mathsf{A}}} % arbitrary PCA
\newcommand{\C}{\ensuremath{\mathcal{C}}} % arbitrary category
\renewcommand{\P}{\ensuremath{{P}}} % arbitrary poset (replaces paragraph symbol)
\renewcommand{\L}{\ensuremath{{L}}} % arbitrary lattice
\newcommand{\X}{\ensuremath{X}} % arbitrary sheaf
\newcommand{\E}{\mathcal{E}} % arbitrary topos
\newcommand{\Ej}{\mathcal{E}_\IBoxSym} % arbitrary topos of sheaves

\newcommand{\Par}[1]{{#1}_\bot}
\newcommand{\ParTo}{\rightharpoonup}
\newcommand{\ParCat}[1]{{#1}^\ParTo}

\newcommand{\PresheafCat}{\hat{\C}}

\newcommand{\Sub}[1]{\mathsf{Sub}({#1})}

%%%%%%%%%%%%%%%%%%%%%%%%%%%%%%%%%%%%%%%%%%%%%%%%%%%%%%%%%%%%%%%%%%%%%%%%%%%%%%%%
%% VERNACULAR COMMANDS %%%%%%%%%%%%%%%%%%%%%%%%%%%%%%%%%%%%%%%%%%%%%%%%%%%%%%%%%
%%%%%%%%%%%%%%%%%%%%%%%%%%%%%%%%%%%%%%%%%%%%%%%%%%%%%%%%%%%%%%%%%%%%%%%%%%%%%%%%

\newcommand{\SystemT}{\textsc{System T}}
\newcommand{\MLTT}{\textsc{MLTT}}
\newcommand{\Agda}{\textsc{Agda}}
\newcommand{\Boxtt}{\textsc{BoxTT}}

%%%%%%%%%%%%%%%%%%%%%%%%%%%%%%%%%%%%%%%%%%%%%%%%%%%%%%%%%%%%%%%%%%%%%%%%%%%%%%%%
%% DECORATION %%%%%%%%%%%%%%%%%%%%%%%%%%%%%%%%%%%%%%%%%%%%%%%%%%%%%%%%%%%%%%%%%%
%%%%%%%%%%%%%%%%%%%%%%%%%%%%%%%%%%%%%%%%%%%%%%%%%%%%%%%%%%%%%%%%%%%%%%%%%%%%%%%%

\setlength{\ULdepth}{3pt}
\contourlength{1pt}
\newcommand{\underlinehelper}[3]{
  \colorbox{#2}{
  #1{\phantom{#3}}%
  \llap{\contour{#2}{#3}}%
  }
}
\newcommand{\hla}[1]{\underlinehelper{\uline}{pink}{#1}}
\newcommand{\hlb}[1]{\underlinehelper{\uwave}{yellow}{#1}}
\newcommand{\hlc}[1]{\underlinehelper{\uuline}{pink}{#1}}

%%%%%%%%%%%%%%%%%%%%%%%%%%%%%%%%%%%%%%%%%%%%%%%%%%%%%%%%%%%%%%%%%%%%%%%%%%%%%%%%
%% EXTERNAL %%%%%%%%%%%%%%%%%%%%%%%%%%%%%%%%%%%%%%%%%%%%%%%%%%%%%%%%%%%%%%%%%%%%
%%%%%%%%%%%%%%%%%%%%%%%%%%%%%%%%%%%%%%%%%%%%%%%%%%%%%%%%%%%%%%%%%%%%%%%%%%%%%%%%

%% Eval map
\newcommand{\EEval}{\mathsf{ev}}

%%%%%%%%%%%%%%%%%%%%%%%%%%%%%%%%%%%%%%%%%%%%%%%%%%%%%%%%%%%%%%%%%%%%%%%%%%%%%%%%
%% INTERNAL %%%%%%%%%%%%%%%%%%%%%%%%%%%%%%%%%%%%%%%%%%%%%%%%%%%%%%%%%%%%%%%%%%%%
%%%%%%%%%%%%%%%%%%%%%%%%%%%%%%%%%%%%%%%%%%%%%%%%%%%%%%%%%%%%%%%%%%%%%%%%%%%%%%%%

% Logical language

\newcommand{\ITop}{\top}

\newcommand{\IBot}{\bot}

\newcommand{\INotSym}{\lnot}
\newcommand{\INot}[1]{\INotSym{#1}}

\newcommand{\IAndSym}{\land}
\newcommand{\IAnd}[2]{{#1}\IAndSym{#2}}

\newcommand{\IOrSym}{\lor}
\newcommand{\IOr}[2]{{#1}\IOrSym{#2}}

\newcommand{\IImpliesSym}{\Rightarrow}
\newcommand{\IImplies}[2]{{#1}\IImpliesSym{#2}}

\newcommand{\IForallSym}{\forall}
\newcommand{\IForall}[2]{\IForallSym{#1}.\,{#2}}

\newcommand{\IExistsSym}{\exists}
\newcommand{\IExists}[2]{\IExistsSym{#1}.\,{#2}}

\newcommand{\IBoxSym}{\Box}
\newcommand{\IBox}[1]{\IBoxSym{#1}}

\newcommand{\ISetComp}[2]{\{{#1}\mid{#2}\}}

\newcommand{\IEqSym}{=}
\newcommand{\IEq}[2]{{#1}\IEqSym{#2}}

% Types
\newcommand{\IPower}[1]{\mathcal{P}{#1}}
\newcommand{\IPowerCl}[1]{\mathcal{P}_{\IBoxSym}{#1}}
\newcommand{\IProp}{\Omega}
\newcommand{\IPropCl}{\Omega_\IBoxSym}
\newcommand{\IHolds}[1]{|{#1}|}

%% Universe
\newcommand{\IUni}{\mathcal{U}}

%% Empty type
\newcommand{\IEmpty}{\mathbb{0}}
\newcommand{\IEmptyM}[1]{!_{#1}}

%% Unit type
\newcommand{\IUnit}{\mathbb{1}}
\newcommand{\IUnitM}[1]{{\langle\rangle}_{#1}}

%% Product types
\newcommand{\IProd}[2]{{#1}\times{#2}}
\newcommand{\IFst}{\pi_0}
\newcommand{\ISnd}{\pi_1}
\newcommand{\IPair}[2]{\langle{#1},{#2}\rangle} % pairing operator
\newcommand{\IPairBi}[2]{{#1}\times{#2}} % bifunctor action of products

%% Equaliser types
\newcommand{\IEqualiser}[2]{\mathsf{Eq({#1,#2})}}

%% Function types
\newcommand{\IFun}[2]{{{#1}\to{#2}}}
\newcommand{\ILam}[2]{{\mathop{\lambda({#1}).}{#2}}}

\newcommand{\IId}[1]{\mathbb{1}_{#1}}

% INTERNAL PCA

%% Partial application
\newcommand{\IApSym}{\cdot}
\newcommand{\IAp}[2]{{#1}\IApSym{#2}}

%% Polynomial application
\newcommand{\IPApSym}{\cdot}
\newcommand{\IPAp}[2]{{#1}\IPApSym{#2}}

%% Polynomial abstraction
\newcommand{\IAbs}[2]{\Lambda{#1}.\,{#2}}

%% Partial element ordering
\newcommand{\IPLeqSym}{\preccurlyeq}
\newcommand{\IPLeq}[2]{{#1}\IPLeqSym{#2}}

\newcommand{\IPSimSym}{\simeq}
\newcommand{\IPSim}[2]{{#1}\IPSimSym{#2}}

\newcommand{\IIdentCode}{\texttt{id}}

\newcommand{\ICompCode}{\texttt{comp}}

%% Pairing
\newcommand{\IPairCode}{\texttt{pair}}
\newcommand{\IFstCode}{\texttt{p}_0}
\newcommand{\ISndCode}{\texttt{p}_1}

%% Disjoint unions
\newcommand{\ILeftCode}{\texttt{in}_0}
\newcommand{\IRightCode}{\texttt{in}_1}
\newcommand{\IMatchCode}{\texttt{match}}


%% Sheafification
\newcommand{\ISheafSym}{\mathcal{D}}
\newcommand{\ISheaf}[1]{\ISheafSym{#1}}
\newcommand{\ISheafExt}[1]{{#1}^\sharp}
\newcommand{\ISheafIndSym}{\mathcal{D}_\text{ind}}
\newcommand{\ISheafInd}[1]{\ISheafIndSym\,{#1}}

\newcommand{\IGlueSym}{\text{glue}}
\newcommand{\IGlue}[1]{\IGlueSym_{#1}}

\newcommand{\ILeafSym}{\eta}
\newcommand{\ILeaf}[1]{\eta\,{#1}}

\newcommand{\IBranchSym}{\beta}
\newcommand{\IBranch}[3]{\beta(#1,#2,#3)}

%% Lifting monad

\newcommand{\IPar}[1]{{#1}_\bot}
\newcommand{\ITotal}[1]{{{#1}\!\downarrow}}
\newcommand{\IValSym}{\text{value}}
\newcommand{\IVal}[1]{\IValSym\,{#1}}

\newcommand{\ISquashSym}{{\Downarrow}}
\newcommand{\ISquash}[1]{\ISquashSym\,{#1}}

\newcommand{\IParExtSym}{\Diamond}
\newcommand{\IParExt}[1]{\IParExtSym{#1}}

%%%%%%%%%%%%%%%%%%%%%%%%%%%%%%%%%%%%%%%%%%%%%%%%%%%%%%%%%%%%%%%%%%%%%%%%%%%%%%%%
%% TRIPOS %%%%%%%%%%%%%%%%%%%%%%%%%%%%%%%%%%%%%%%%%%%%%%%%%%%%%%%%%%%%%%%%%%%%%%
%%%%%%%%%%%%%%%%%%%%%%%%%%%%%%%%%%%%%%%%%%%%%%%%%%%%%%%%%%%%%%%%%%%%%%%%%%%%%%%%

%% Tripos
\newcommand{\Trip}{T_\IBoxSym}

%% Realizability Predicates
\newcommand{\RealPred}[1]{\IPowerCl{(\IProd{#1}{\A})}}

% Tripos language

%% Helper macro to tag tripos language symbols with the object type
\newcommand{\TagWithObject}[2]{\overset{\scriptscriptstyle #2}{#1}}

\newcommand{\TTop}[1]{\TagWithObject{\top}{#1}}

\newcommand{\TBot}[1]{\TagWithObject{\bot}{#1}}

\newcommand{\TAndSym}[1]{\TagWithObject{\land}{#1}}
\newcommand{\TAnd}[3]{{#1}\TAndSym{#3}{#2}}

\newcommand{\TOrSym}[1]{\TagWithObject{\lor}{#1}}
\newcommand{\TOr}[3]{{#1}\TOrSym{#3}{#2}}

\newcommand{\TImpliesSym}[1]{\TagWithObject{\Rightarrow}{#1}}
\newcommand{\TImplies}[3]{{#1}\TImpliesSym{#3}{#2}}

\newcommand{\TPull}[1]{{#1}^\ast}
\newcommand{\TExists}[1]{\exists_{#1}}
\newcommand{\TForall}[1]{\forall_{#1}}

%% Category of assemblies
\newcommand{\AsmSym}{\mathbf{Asm}}
\newcommand{\Asm}[3]{\AsmSym_{(#2,#3)}(#1)}
\newcommand{\AsmCat}{\Asm{\A}{\E}{\Box}}

\newcommand{\Adj}[2]{{#1}[{#2}]}


%--------------------------------------------------------------------------------
% Document proper
\begin{document}

\title{Realizability with Sheaves}
\author{Bruno da Rocha Paiva}
\maketitle

We begin a treatment of realizability internal to categories of sheaves.

%\section{Internal Language of a Presheaf Topos}
%
%A typical course in mathematical logic may start by looking at formal systems
%such as natural deduction or the sequent calculus, giving rise to a notion of
%truth, namely the provability of formulas. A natural next step is to then look
%at the semantics of said proof system. These semantics can vary significantly
%depending on the chosen logic such as boolean valuations for propositional
%logic, Kripke frames for modal logic, \(\mathcal{L}\)-structures for predicate
%logic, etc. Regardless, in each case we can consider whether a formula is valid,
%i.e.\ true in every model and this gives rise to yet to a second notion of
%truth. Remarkably, these two notions of truth often coincide, provided the right
%proof system and semantics are chosen. While provability quantifies
%existentially over proofs, asking whether a formal proof exists, validity is a
%universal statement asking that the interpretation of a formula hold in all
%models. This duality is powerful: to show a formula \(\phi\) is not provable by
%proof-theoretic means is difficult. Typical methods start by identifying some
%property of proofs \(P\) such that any proof of any formula \(\psi\) can be
%transformed to another proof of \(\psi\) satisfying \(P\). From here, if we can
%show that no proof of \(\phi\) satisfying \(P\) exists, then that must mean no
%proof exists in general. Typical examples of properties \(P\) can be that the
%proof does not make use of the \textbf{Cut} rule in sequent calculus or
%alternatively for natural deduction that there exist no \(\beta\)-redexs. While
%these are interesting and useful results to have, in this case the semantics
%offers an easier route. To show \(\phi\) is not valid it suffices to demonstrate
%a model in which the interpretation of \(\phi\) does not hold and this is
%generally easier than the syntactic route.
%
%At a first glance the power of this duality seems to be in dispatching of syntax
%and work purely with models. As we move to more and more complicated languages,
%going beyond first-order logic to its higher-order counterparts, then the needle
%begins to move back. In particular, consider (your favourite version of)
%Martin-L\"of Type Theory (\MLTT{}) with \(\Pi\)-types and \(\Sigma\)-types.
%Semantics for \MLTT{} are usually categorically, either through categories with
%families, contextual categories, natural models, the list goes on. Fixing your
%favourite notion of model, categories of presheaves \(\PresheafCat\) generally
%give rise to a model of dependent type theory. That means if we prove some
%statement in \MLTT{}, then by taking the interpretation of said statement in
%\(\PresheafCat\) we will get a true statement about presheaves over \(\C\).
%Working with presheaves can be quite labourious and often becomes an exercise
%in bookkeeping of indices.
%
%\subsection{Exponentials}
%
%Given two types \(A, B\), the type of functions from \(A\) to \(B\),
%denoted \(\IFun{A}{B}\), behaves like the usual function type in type theory.
%%
%As for its categorical semantics we have the following
%
%\begin{defn}
%  Given two objects \(A, B\), their exponential is an object \(\IFun{A}{B}\)
%  along with a morphism \(\EEval : A \times \IFun{A}{B} \to B\) such that
%  for all \(f : C \times A \to B\) there exists a unique
%  \(\tilde{f}: C \to \IFun{A}{B}\) making the following triangle commutes
%  %
%  \[\begin{tikzcd}
%    {C \times \IFun{A}{B}} && B \\
%    \\
%    {C\times A}
%    \arrow["{\EEval}", from=1-1, to=1-3]
%    \arrow["{C\times\tilde{f}}", from=3-1, to=1-1]
%    \arrow["f"', from=3-1, to=1-3]
%  \end{tikzcd}\]
%\end{defn}
%
%\subsubsection{Exponentials of Presheaves}
%
%The exponential of two presheaves \(X,Y\) is a presheaf
%\(\IFun{X}{Y}\) with action on objects given by
%%
%\[
%  (\IFun{X}{Y})_{w} \IsDefined \mathsf{Nat}(\yo(w) \times X, Y)
%\]
%%
%and action on morphisms defined as follows
%%
%\[\begin{array}{l}
%  \IFun{X}{Y}_{f:u \to w} :
%    \mathsf{Nat}(\yo(w) \times X, Y) \to \mathsf{Nat}(\yo(u) \times X, Y)\\
%  (\IFun{X}{Y}_{f:u \to w}(\alpha))_{v}(g : v \to u, x : X_{v})
%    \IsDefined \alpha_{v}(fg, x)
%\end{array}\]
%
%The evaluation map is a natural transformation
%\(\EEval : X \times \IFun{X}{Y} \to Y\) defined for each object
%\(w\) by
%%
%\[\begin{array}{l}
%  \EEval_{w} : X \times \mathsf{Nat}(\yo(w) \times X, Y) \to Y\\
%  \EEval_{w}(x, \alpha) \IsDefined \alpha_{w}(\mathsf{id}, x)
%\end{array}\]
%%
%Naturality of \(\EEval\) means for all \(x : X_{w}\),
%\(\alpha : \mathsf{Nat}(\yo(w) \times X, Y)\) and \(f : v \to w\) we have
%\[
%  Y_{f}(\alpha_{w}(\mathsf{id}, x)) =
%  Y_{f}(\EEval_{w}(x, \alpha)) =
%  \EEval_{v}(X_{f}(x), \IFun{X}{Y}_{f}(\alpha)) =
%  (\IFun{X}{Y}_{f}(\alpha))_{v}(\mathsf{id}, X_{f}(x)) =
%  \alpha_{v}(f, X_{f}(x))
%\]
%
%\subsection{Substitution and Logical Quantification}
%
%Any topos is able to interpret both proof-relevant and proof-irrelevant
%universal and existential quantification.
%%
%Let us focus on proof-irrelevant quantification, that is quantification yielding
%propositions.
%%
%For any map \(f : X \to Y\) we have a reindexing map
%\(f^{*} : \mathsf{Sub}(Y) \to \mathsf{Sub}(X)\) by precomposition with \(f\).
%%
%This reindexing preserves all Heyting algebra structure and gives our
%interpretation of substitution of terms in predicates.
%%
%As we are in a topos, we also happen to have left and right adjoints
%\(\exists_{f} \dashv f^{*} \dashv \forall_{f}\), though this time they are only
%monotone maps and do not preserve all the Heyting algebra structure.
%%
%These adjoints can be shown to satisfy the Beck-Chevalley conditions, so they
%form, respectively, our interpretation of existential and universal quantifiers.
%
%Focusing on existential quantification first, suppose we have a predicate
%\(\phi : \mathsf{Sub}(X \times Y)\) in two typed variables \(x\) and \(y\).
%%
%Considering the projection \(\pi : X \times Y \to X\), we can get a predicate
%\(\exists_{\pi}\phi : \mathsf{Sub}(X)\).
%%
%Abusing some notation, if initially we saw \(\phi\) as the interpretation of
%some syntactical predicate \(\phi(x,y)\), then we would see this new predicate
%as the interpretation of the predicate \(\exists y : Y, \phi(x,y)\)
%with a single free variable \(x\).
%%
%In fact, from adjointness gives us the expected introduction and elimination
%rules.
%%
%Suppose we have a term \(t(x)\) of type \(Y\) such that we can prove
%\(\phi(x,t(x))\).
%%
%From the denotation of \(t(x)\) we expect to get some morphism \(X \to Y\)
%which we can get extend to a section of \(\pi : X \times Y \to X\)
%
%some kind of morphism \(\llbracket t(x) \rrbracket : X \to Y\)
%
% for some term \(t(x)\), then
%from said term we expect to
%
%\subsubsection{Quantifiers in Presheaves}
%
%Fix a map \(\phi : X \to Y\) of presheaves and a subpresheaf
%\(S \hookrightarrow X\), with which we can look at both types of quantifiers.
%
%For existentials, we apply \(\exists_{\phi}\) to \(S\) giving us a subpresheaf
%of \(Y\) which is described by
%%
%\[
%  {(\exists_{\phi}S)}_{w} =
%  \{y : Y_{w} \mid \exists x \in S_{w}, \phi_{w}(x) = y\} \subseteq Y_{w}
%\]
%%
%or equivalently described by
%%
%\[
%  {(\exists_{\phi}S)}_{w} =
%  \{y : Y_{w} \mid
%  \forall f : v \to w, \exists x \in S_{v}, \phi_{v}(x) = Y_{f}(y) \}
%\]
%%
%as if we have \(x \in S_{w}\) such that \(\phi_{w}(x)=y\) then, by naturality of
%\(\phi\), we can see that for all \(f : v \to w\) we have
%\(\phi_{v}(X_{f}(x))=Y_{f}(\phi_{w}(x))=Y_{f}(y)\) meaning that \(X_{f}(x)\)
%witnesses \(\exists x \in S_{v}, \phi_{v}(x) = Y_{f}(y)\).
%
%For universals, applying \(\forall_{\phi}\) to \(S\) gives the following
%subpresheaf
%%
%\[
%  {(\forall_{\phi}S)}_{w} =
%  \{y : Y_{w} \mid
%  \forall f : v \to w, \forall x \in X_{v},
%  \phi_{w}(x) = y \implies x \in S_{v} \} \subseteq Y_{w}
%\]
%%
%Unlike for the existential quantifier, there is no simplification allowing
%us to remove quantification over morphisms.
%%
%This is due to the fact that a subpresheaf must be monotonic, hence
%the similarity to Kripke semantics of universal quantification.




%\newpage
%\section{Internal PCAs}
%
%In set-based mathematics, partial functions \(f : X \ParTo Y\) are usually
%defined as a subset \(\text{dom}(f) \subseteq X\) and a total function
%\(f : \text{dom}(f) \to Y\) on said subset. With such a definition, the
%composition of two partial functions \(f : X \ParTo Y\) and \(g : Y \ParTo Z\)
%will have domain \(\text{dom}(g \circ f) = \text{im}(f) \cap \text{dom}(g)\), on
%which the composite \(g \circ f\) is total. Associativity of function
%composition comes down to associativity of subset intersections and as a total
%function is ``trivially'' partial we also have identity maps satisfying the unit
%laws. With this, we have just defined a category of sets and partial functions
%between them. A similar construction can be done for arbitrary categories under
%the mild condition that pullbacks along monomorphisms exist. Said pullbacks will
%provide composition of partial morphisms, leading us to the final obstruction,
%which is that pullbacks are only associative up to (unique) isomorphism. This is
%remedied by identifying partial maps up to isomorphism of the domains as we will
%make explicit later.
%
%\begin{defn}
%  Given a category \(\C\) with pullbacks we define the \definiendum{category of
%    partial maps} in \(\C\), denoted \(\ParCat{\C}\), according to the following:
%  \begin{itemize}
%    \item objects of \(\ParCat{\C}\) are exactly those of \(\C\).
%    \item given objects \(X,Y\), their hom-set \(\ParCat{\C}(X,Y)\) is given by
%      equivalence classes of spans \(X \xhookleftarrow{m} S \xrightarrow{f} Y\)
%      with \(m\) monic. Two such spans
%      \(X \xhookleftarrow{m} S \xrightarrow{f} Y\) and
%      \(X \xhookleftarrow{n} P \xrightarrow{g} Y\) are equivalent if there
%      exists an isomorphism \(f : S \cong P\) such that the following commutes
%      \[\begin{tikzcd}
%          && S \\
%          X &&&& Y \\
%          && P \arrow["m"', hook, from=1-3, to=2-1] \arrow["f", from=1-3, to=2-5] \arrow["f"', from=1-3, to=3-3] \arrow["n", hook', from=3-3, to=2-1] \arrow["g"', from=3-3, to=2-5]
%        \end{tikzcd}\]
%    \item Composition of two morphisms
%      \(X \xhookleftarrow{m} S \xrightarrow{f} Y\) and
%      \(Y \xhookleftarrow{n} P \xrightarrow{g} Z\) is given by the pullback
%      \[\begin{tikzcd}
%          X \\
%          S & Y \\
%          {S\times_YP} & P & Z \arrow["m", hook', from=2-1, to=1-1] \arrow["f"', from=2-1, to=2-2] \arrow[from=3-1, to=2-1] \arrow["\lrcorner"{anchor=center, pos=0.125, rotate=90}, draw=none, from=3-1, to=2-2] \arrow[from=3-1, to=3-2] \arrow["n", hook', from=3-2, to=2-2] \arrow["g"', from=3-2, to=3-3]
%        \end{tikzcd}\]
%    \item Identity maps are given by the span consisting of two identity maps.
%  \end{itemize}
%\end{defn}
%
%The associative and unit laws can be checked to follow from the uniqueness (up
%to isomorphism) of pullbacks. Working through this definition for the category
%of sets we recover a category isomorphic to our original description. In the
%specific case of sets we were able to avoid worrying about equivalence classes
%by using subsets instead of subobjects and subset intersections instead of
%pullbacks. Similarly, we can give a more concrete description for the category
%of partial maps of sheaves.

\section{A Realizability Tripos}

In this section we fix a topos \(\mathcal{E}\) along with a Lawvere-Tierney
topology \(\Box : \Omega \to \Omega\) on this topos.
%
I believe for this definition \(\mathcal{E}\) can be an elementary topos,
but I am less sure how this works out once we start considering sheafification.
%
We give our definitions internally to \(\mathcal{E}\), letting us simplify
the proofs in this section.


\begin{defn}\label{defn:tripos-predicates-and-ordering}
  Given a presheaf \(\X\) we define the type of \definiendum{realizability
    predicates on \(\X\)} as the exponential \(\RealPred{\X} \).
  %
  We further define an ordering on realizability
  predicates~\(\varphi, \psi : \RealPred{\X}\) by
  %
  \[
    \varphi \leq \psi
    \IsDefined
    \IExists{e : \A}{
      \IForall{x : \X}{
        \IForall{a : \A}{
          \IImplies%
            {\varphi(a,x)}%
            {\IBox{(
              \IAnd%
              {\ITotal{\IAp{e}{a}}}%
              {\psi(\IAp{e}{a}, x)}
            )}}
        }
      }
    }
  \]
  %
  We will say that \definiendum{\(e\) evidences \(\varphi \leq \psi\)} if
  it witnesses the existential quantifier above.
\end{defn}

\begin{lemm}\label{lemm:tripos-preorder}
  For any presheaf \(\X\), \((\RealPred{\X}, \leq)\) is a preorder.
\end{lemm}
\begin{proof}
  Fix \(\varphi\), then \(\varphi \leq \varphi\) is always evidenced by
  \(\IIdentCode\).
  %
  Fix \(x : \X\) and \(a \in \varphi(x)\), since application of \(\IIdentCode\) is
  always well-defined and behaves as identity, it holds
  that~\(\ITotal{\IAp{\IIdentCode}{a}}\) and \({\IAp{\IIdentCode}{a}\in\varphi(x)}\).
  %
  This implies the necessary boxed statement to prove \(\leq\) reflexive.

  It is also transitive. Suppose that \(\varphi \leq \psi\) and
  \(\psi \leq \chi\) evidenced by \(e_{1}\) and \(e_{2}\) respectively.
  %
  Fixing \(x : \X\) and \(a \in \varphi(x)\), it follows by assumption that
  %
  \[
    \IBox{(
      \IAnd%
      {\ITotal{\IAp{e_{1}}{a}}}%
      {\IAp{e_{1}}{a} \in \psi(x)}
    )}
  \]
  %
  By monotonicity of \(\IBoxSym\) as well as our second assumption we may derive
  %
  \[
    \IBox{(
      \IAnd%
      {\ITotal{\IAp{e_{1}}{a}}}%
      {\IBox{[\IAnd%
        {\ITotal{\IAp{e_{2}}{(\IAp{e_{1}}{a})}}}%
        {\IAp{e_{2}}{(\IAp{e_{1}}{a})} \in \chi(x)}
      ]}}
    )}
  \]
  %
  As \(\IBoxSym\) is idempotent and preserves conjunctions we may simplify this
  to contain a single occurence of the modal box
  %
  \[
    \IBox{(
      \IAnd%
      {\ITotal{\IAp{e_{1}}{a}}}%
      {\IAnd%
        {\ITotal{\IAp{e_{2}}{(\IAp{e_{1}}{a})}}}%
        {\IAp{e_{2}}{(\IAp{e_{1}}{a})} \in \chi(x)}
      }
    )}
  \]
  %
  As needed, this is equivalent to saying that
  \(\IAp{\IAp{\ICompCode}{e_{1}}}{e_{2}}\) evidences \(\varphi \leq \chi\).
\end{proof}

\begin{lemm}\label{lemm:tripos-finite-meets}
  For any presheaf \(\X\), the preorder \((\RealPred{\X}, \leq)\) has finite
  meets.
  %
  The top element is described by~\ref{defn:tripos-top} and the meet of two
  realizability predicates~\(\varphi\) and \(\psi\) is described
  by~\ref{defn:tripos-binary-meet}:
  %
  \begin{gather}
    \TTop{\X}(x,a)
    \IsDefined
    \ITop
    \label{defn:tripos-top}
    \\
    (\TAnd{\varphi}{\psi}{\X})(x,a)
    \IsDefined
    \IAnd%
      {\IAnd{\ITotal{\IAp{\IFstCode}{a}}}{\varphi(x,\IAp{\IFstCode}{a})}}%
      {\IAnd{\ITotal{\IAp{\ISndCode}{a}}}{\psi(x,\IAp{\ISndCode}{a})}}
    \label{defn:tripos-binary-meet}
  \end{gather}
\end{lemm}
\begin{proof}
  We start by showing that \(\TTop{\X}\) is a greatest element.
  %
  Fixing \(\varphi:\RealPred{\X}\), then the identity code \(\IIdentCode\)
  evidences \(\varphi \leq \TTop{\X}\), as for all \(x : \X\) and
  \(a \in \varphi(x)\) we have \(\IAp{\IIdentCode}{a} = a\).

  Next we show \(\TAnd{\varphi}{\psi}{\X}\) is the meet of \(\varphi\) and
  \(\psi\) in \(\RealPred{\X}\).
  %
  The projection codes~\(\IFstCode\) and~\(\ISndCode\) evidence
  \(\TAnd{\varphi}{\psi}{\X}\leq\varphi\) and
  \(\TAnd{\varphi}{\psi}{\X}\leq\psi\) respectively:
  %
  if \((\TAnd{\varphi}{\psi}{\X})(x,a)\) then by definition we have
  \(\IAnd{\ITotal{\IAp{\IFstCode}{a}}}{\varphi(x,\IAp{\IFstCode}{a})}\), as
  well as the corresponding for \(\ISndCode\) and \(\psi\).
  %
  As for the elimination rule, suppose we have \(\chi\leq\varphi\) and
  \(\chi\leq\psi\) evidenced by \(e_{1}\) and \(e_{2}\) respectively,
  then~%
  %
  \(
    e \IsDefined
    \IAbs{a}{\IPAp{\IPAp{\IPairCode}{(\IPAp{e_{1}}{a})}}{(\IPAp{e_{1}}{a})}}
  \)
  %
  evidences \(\chi\leq\TAnd{\varphi}{\psi}{\X}\).
  %
  We know this by specialising our assumptions on~\(e_{1}\) and~\(e_{2}\) to a
  fixed \(x:\X\) and \(a:\A\) satisfying \(\chi(x,a)\), and commuting
  \(\IBoxSym\) with conjunctions, giving us
  %
  \[
    \IBox{
      (\IAnd{
        \IAnd{\ITotal{\IAp{e_{1}}{a}}}{\ITotal{\IAp{e_{2}}{a}}}
      }{
        \IAnd{\varphi(x,\IAp{e_{1}}{a})}{\psi(x,\IAp{e_{2}}{a})}
      })
    }
  \]
  %
  The first two conjuncts tell us that \(\ITotal{\IAp{e}{a}}\) and so by the
  specification of functional completeness we see that
  \(\IAp{e}{a} = \IAp{\IAp{\IPairCode}{(\IAp{e_{1}}{a})}}{(\IAp{e_{1}}{a})}\).
\end{proof}

\begin{lemm}\label{lemm:tripos-finite-joins}
  For any presheaf \(\X\), the preorder \((\RealPred{\X}, \leq)\) has finite
  joins.
  %
  The bottom element is described by~\ref{defn:tripos-bottom} and the join of two
  realizability predicates~\(\varphi\) and \(\psi\) is described
  by~\ref{defn:tripos-binary-join}:
  %
  \begin{gather}
    \TBot{\X}(x,a)
    \IsDefined
    \IBot
    \label{defn:tripos-bottom}
    \\
    (\TOr{\varphi}{\psi}{\X})(x,a)
    \IsDefined
    \IExists{b:\A}{\IOr%
      {(\IAnd{\IEq{a}{\IAp{\ILeftCode}{b}}}{\varphi(x,b)})}%
      {(\IAnd{\IEq{a}{\IAp{\IRightCode}{b}}}{\psi(x,b)})}
    }
    \label{defn:tripos-binary-join}
  \end{gather}
\end{lemm}
\begin{proof}
  First we see that \(\TBot{\X}\) is indeed the bottom element, but any code
  \(e : \A\) vacuously evidences the inequality \(\TBot{\X} \leq \phi\).

  Now let us see that \(\TOr{\varphi}{\psi}{\X}\) is indeed the required join.
  %
  The introduction rules \(\varphi\leq\TOr{\varphi}{\psi}{\X}\) and
  \(\psi\leq\TOr{\varphi}{\psi}{\X}\) are evidenced by \(\ILeftCode\) and
  \(\IRightCode\) respectively.
  %
  To prove the former, fix \(x:\X\) and \(a:\A\) such that \(\varphi(x,a)\),
  then we always have \(\ITotal{\IAp{\ILeftCode}{a}}\) and hence
  \((\TOr{\varphi}{\psi}{\X})(x,\IAp{\ILeftCode}{a})\).
  %
  \(\Box\) is inflationary so this finishes the proof and we note that the case
  for \(\IRightCode\) follows similarly.
  %
  As for the elimination rule, suppose that \(e_{0}\) evidences
  \(\varphi \leq \chi\) and \(e_{1}\) evidences \(\psi \leq \chi\), then
  \(\IAp{\IAp{\IMatchCode}{e_{0}}}{e_{1}}\) will evidence
  \(\TOr{\varphi}{\psi}{\X} \leq \chi\).
  %
  In fact, fix \(x:\X\) and \(a:\A\) such that
  \((\TOr{\varphi}{\psi}{\X})(x,a)\), then there exists some \(b:\A\) such that
  either \(\IEq{a}{\IAp{\ILeftCode}{b}}\) and \(b\in\varphi(x)\), or
  \(\IEq{a}{\IAp{\IRightCode}{b}}\) and \(b\in\psi(x)\).
  %
  In the first case, we have that
  %
  \(\IPLeq%
  {\IAp{e_{0}}{b}}%
  {\IAp{\IAp{\IAp{\IMatchCode}{e_{0}}}{e_{1}}}{(\IAp{\IRightCode}{b})}}%
  \)
  %
  and \(\IBox{(\IAnd{\ITotal{\IAp{e_{0}}{b}}}{\chi(\IAp{e_{0}}{b},x)})}\),
  from which we can show that
  %
  \(\IBox{(\IAnd%
    {\ITotal{\IAp{\IAp{\IAp{\IMatchCode}{e_{0}}}{e_{1}}}{a}}}%
    {\chi(\IAp{\IAp{\IAp{\IMatchCode}{e_{0}}}{e_{1}}}{a},x)}%
  )}\).
  %
  The right case follows analogously.
\end{proof}

\begin{lemm}\label{lemm:tripos-heyting-implication}
  For any presheaf \(\X\), \((\RealPred{\X}, \leq)\) has Heyting
  implication described by
  \begin{gather}
    (\TImplies{\varphi}{\psi}{\X})(x,a)
    \IsDefined
    \IForall{b : \A}{%
      \IImplies%
      {\varphi(x,b)}%
      {\IBox{(%
          \IAnd%
          {\ITotal{\IAp{a}{b}}}%
          {\psi(x,\IAp{a}{b})}
    )}}}
    \label{defn:tripos-heyting-implication}
  \end{gather}
\end{lemm}
\begin{proof}
  Suppose that \(\TAnd{\varphi}{\psi}{\X} \leq \chi\) is evidenced by \(e\),
  then \(\varphi \leq \TImplies{\psi}{\chi}{\X}\) is realized by the code
  \(\IAbs{a}{\IAbs{b}{\IPAp{e}{(\IPAp{\IPAp{\IPairCode}{a}}{b})}}}\).
  %
  Fixing \(x : \X\) and \(a : \A\) such that \(\varphi(x,a)\), we have
  %
  \[\IEq%
  {\IAp{(\IAbs{a}{\IAbs{b}{\IPAp{e}{(\IPAp{\IPAp{\IPairCode}{a}}{b})}}})}{a}}%
  {\IAbs{b}{\IPAp{e}{(\IPAp{\IPAp{\IPairCode}{a}}{b})}}}%
  \]
  %
  leaving us to prove that
  %
  \(
  (\TImplies{\psi}{\chi}{\X})%
  (x, \IAbs{b}{\IPAp{e}{(\IPAp{\IPAp{\IPairCode}{a}}{b})}})%
  \).
  %
  For this, fix \(b : \A\) and assume \(\psi(x,b)\), then we know that
  %
  \(\IPLeq%
  {\IPAp{e}{(\IPAp{\IPAp{\IPairCode}{a}}{b})}}%
  {\IAp{(\IAbs{b}{\IPAp{e}{(\IPAp{\IPAp{\IPairCode}{a}}{b})}})}{b}}\).
  %
  From \(\varphi(x,a)\) and \(\psi(x,b)\) we know that
  \((\TAnd{\varphi}{\psi}{\X})(x,\IPAp{\IPAp{\IPairCode}{a}}{b})\) and hence
  %
  \(\IBox{(%
    \IAnd%
    {\ITotal{\IAp{e}{\IPAp{\IPAp{\IPairCode}{a}}{b}}}}%
    {\psi(x,\IAp{e}{(\IPAp{\IPAp{\IPairCode}{a}}{b})})})}%
  \)
  %
  as needed.

  For the converse direction suppose that
  \(\varphi \leq \TImplies{\psi}{\chi}{\X}\) is evidenced by \(e\), then
  \(\TAnd{\varphi}{\psi}{\X} \leq \chi\) will be evidenced by
  \(\IAbs{a}{\IPAp{\IPAp{e}{(\IPAp{\IFstCode}{p})}}{(\IPAp{\ISndCode}{p})}}\).
  %
  We fix \(x : \X\) and \(a : \A\) and assume \((\TAnd{\varphi}{\psi})(x,a)\),
  which implies \(\ITotal{\IPAp{\IFstCode}{p}}\),
  \(\ITotal{\IPAp{\ISndCode}{p}}\) and both \(\varphi(x,\IPAp{\IFstCode}{p})\)
  and \(\psi(x,\IPAp{\ISndCode}{p})\).
  %
  From this we get that
  %
  \[
    \IBox{[\IAnd%
    {\ITotal{\IPAp{e}{(\IPAp{\IFstCode}{p})}}}%
    {(\TImplies{\psi}{\chi}{\X})(x,\IPAp{e}{(\IPAp{\IFstCode}{p})})}%
    ]}
  \]
  %
  but \((\TImplies{\psi}{\chi}{\X})(x,\IPAp{e}{(\IPAp{\IFstCode}{p})})\) and
  \(\varphi(x,\IPAp{\IFstCode}{p})\) together imply
  %
  \[
    \IBox{[\IAnd%
    {\ITotal{\IPAp{\IPAp{e}{(\IPAp{\IFstCode}{p})}}{(\IPAp{\ISndCode}{p})}}}%
    {\chi(x,\IPAp{\IPAp{e}{(\IPAp{\IFstCode}{p})}}{(\IPAp{\ISndCode}{p})})}%
    ]}
  \]
  and so monotonicity and idempotence of \(\IBoxSym\) let us string
  these two implications together.
\end{proof}

\todo[inline,caption={}]{
 type up the following from notes:
 \begin{itemize}
   \item define pullback and show it is heyting algebra morphism
   \item prove left adjoint to pullback exists (as monotone map)
   \item prove right adjoint to pullback exists (as monotone map)
   \item prove beck-chevalley condition
   \item show there is a generic object
 \end{itemize}
}

\subsection{The need for a modality}

One approach we could try would be to keep the typical ordering
used in realizability, that is without a modality:
%
\[
    \varphi \leq' \psi
    \IsDefined
    \IExists{e : \A}{
      \IForall{x : \X}{
        \IForall{a : \A}{
          \IImplies%
            {\varphi(a,x)}%
            {\IAnd%
              {\ITotal{\IAp{e}{a}}}%
              {\psi(\IAp{e}{a}, x)}
            }
        }
      }
    }
\]
%
From here we would hope to find an appropriate monad \(T : \AsmCat \to \AsmCat\)
on assemblies and use its Kleisli category for semantics.

Given an assembly \((X,\varphi)\), denote its underlying
object of \(T(X,\varphi)\) by \(X_{T}\) and its realizability relation
by \(\varphi_{T}:\RealPred{X_{T}}\).
%
With this notation fixed, a morphism between \((X,\varphi)\) and \((Y,\psi)\) in
the Kleisli category of \(T\) would then consist of a morphism
\(f : X \to Y_{T}\) in the underlying category \(\E\) satisfying
\(\varphi \leq' \TPull{f}\psi\).
%
This requirement on \(f\) unfolds to
%
\[
  \IExists{e : \A}{
    \IForall{x : \X}{
      \IForall{a : \A}{
        \IImplies%
          {\varphi(a,x)}%
          {\IAnd%
            {\ITotal{\IAp{e}{a}}}%
            {\psi_{T}(\IAp{e}{a}, f(x))}
          }
      }
    }
  }
\]
%
What becomes clear now is that the requirement that \(\IAp{e}{a}\)
be defined sits outside the purview of the monad \(T\)
%
In particular, regardless of the effect on \(X_{T}\) and \(\varphi_{T}\), we
still require \(\IsTotal{\IAp{e}{a}}\) with no alterations.
%
In the presheaf case this amounts to requiring that \({e}\cdot_{w}{a}\)
be defined, instead of asking that there exist a bar \(U\) of \(w\)
such that \({e}\cdot_{u}{a}\) is defined for all \(u \in U\).

\subsection{Alternative Definitions}

We have been very particular with where we use the modality \(\Box\) in the
ordering of realizability predicates, but there are certainly other options.
%
We spend a bit of time now considering said options and their logical strength,
to convince you dear reader that ours is the most canonical.

%\begin{gather}
%    \IExists{e : \A}{
%      \IForall{x : \X}{
%        \IForall{a : \A}{
%          \IImplies%
%            {\varphi(a,x)}%
%            {(
%              \IAnd%
%              {\ITotal{\IAp{e}{a}}}%
%              {\psi(\IAp{e}{a}, x)}
%            )}
%        }
%      }
%    }
%    \\
%    \IExists{e : \A}{
%      \IForall{x : \X}{
%        \IForall{a : \A}{
%          \IImplies%
%            {\varphi(a,x)}%
%            {\IBox{(
%              \IAnd%
%              {\ITotal{\IAp{e}{a}}}%
%              {\psi(\IAp{e}{a}, x)}
%            )}}
%        }
%      }
%    }
%    \\
%    \IExists{e : \A}{
%      \IForall{x : \X}{
%        \IForall{a : \A}{
%          \IImplies%
%            {\IBox{(\varphi(a,x))}}%
%            {(
%              \IAnd%
%              {\ITotal{\IAp{e}{a}}}%
%              {\psi(\IAp{e}{a}, x)}
%            )}
%        }
%      }
%    }
%    \\
%    \IExists{e : \A}{
%      \IForall{x : \X}{
%        \IForall{a : \A}{
%          \IBox{(\IImplies%
%            {\varphi(a,x)}%
%            {(
%              \IAnd%
%              {\ITotal{\IAp{e}{a}}}%
%              {\psi(\IAp{e}{a}, x)}
%            )})
%        }}
%      }
%    }
%    \\
%    \IBox{(\IExists{e : \A}{
%      \IForall{x : \X}{
%        \IForall{a : \A}{
%          \IBox{(\IImplies%
%            {\varphi(a,x)}%
%            {(
%              \IAnd%
%              {\ITotal{\IAp{e}{a}}}%
%              {\psi(\IAp{e}{a}, x)}
%            )})
%        }}
%      }
%    })}
%    \\
%    \IBox{(\IExists{e : \A}{
%      \IForall{x : \X}{
%        \IForall{a : \A}{
%          \IImplies%
%            {\varphi(a,x)}%
%            {(
%              \IAnd%
%              {\ITotal{\IAp{e}{a}}}%
%              {\psi(\IAp{e}{a}, x)}
%            )}}
%      }
%    })}
%\end{gather}

First it is helpful to see how \(\IBoxSym\) interacts with Heyting implication.
%
If we have a single implication \(\IImplies{A}{B}\) then there are 3 positions
in which we can place a \(\IBoxSym\), namely in front of the \(A\), in front
of the \(B\) and surrounding the whole implication.
%
Furthermore, as the modality is idempotent, we only need to consider at most
one box in each position.
%
This leaves us with eight possibilities, four of which turn out to be
equivalent.
%
We display all of these below, with the arrows displaying logical implication.
%
\begin{center}
\begin{tikzpicture}
  \node (a) at (-4.5,3) {\(\IImplies{A}{\IBox{B}}\)};
  \node (b) at (-1.5,3) {\(\IImplies{\IBox{A}}{\IBox{B}}\)};
  \node (c) at (1.5,3) {\(\IBox{(\IImplies{\IBox{A}}{\IBox{B}})}\)};
  \node (d) at (4.5,3) {\(\IBox{(\IImplies{A}{\IBox{B}})}\)};
  \node (e) at (0,2) {\(\IBox{(\IImplies{A}{B})}\)};
  \node (f) at (-1.5,1) {\(\IImplies{A}{B}\)};
  \node (g) at (1.5,1) {\(\IBox{(\IImplies{\IBox{A}}{B})}\)};
  \node (h) at (0,0) {\(\IImplies{\IBox{A}}{B}\)};

  \draw[->] (h) -- (f);
  \draw[->] (h) -- (g);
  \draw[->] (f) -- (e);
  \draw[->] (g) -- (e);
  \draw[->] (e) -- (a);
  \draw[->] (e) -- (b);
  \draw[->] (e) -- (c);
  \draw[->] (e) -- (d);
  \draw[<->] (a) -- (b);
  \draw[<->] (b) -- (c);
  \draw[<->] (c) -- (d);
\end{tikzpicture}
\end{center}
%
\todo[inline]{I think we may also have that \(\IBox{(\IImplies{A}{B})}\) and
  \(\IBox{(\IImplies{\IBox{A}}{B})}\) are equivalent to the top formulas.}
%
%\begin{center}
%\begin{tikzpicture}
%
%  \node (b) at (13,0) {\(\IBox{(\IImplies{\IBox{A}}{\IBox{B}})}\)};
%  \node (d) at (11,1) {\(\IBox{(\IImplies{A}{\IBox{B}})}\)};
%  \node (f) at (11,-1) {\(\IBox{(\IImplies{\IBox{A}}{B})}\)};
%  \node (e) at (8,1) {\(\IBox{(\IImplies{A}{B})}\)};
%  \node (c) at (8,-1) {\(\IImplies{\IBox{A}}{\IBox{B}}\)};
%  \node (a) at (6,0) {\(\IImplies{A}{\IBox{B}}\)};
%
%  \node (g) at (4,0) {\(\IImplies{A}{B}\)};
%
%  \node (h) at (2,0) {\(\IImplies{\IBox{A}}{B}\)};
%
%  \draw[->] (h) -- (g);
%  \draw[->] (g) -- (a);
%  \draw[<->] (a) -- (e);
%  \draw[<->] (e) -- (d);
%  \draw[<->] (d) -- (b);
%  \draw[<->] (b) -- (f);
%  \draw[<->] (f) -- (c);
%  \draw[<->] (c) -- (a);
%\end{tikzpicture}
%\end{center}
%
The usual definition of implication in realizability uses Heyting implication
without any boxes, which turns out to be too strong a statement in our setting.
%
The above shows there is only one way of weakening an implication with
\(\IBoxSym\).
%
The usual definition of implication in realizability uses Heyting implication
without any boxes, which turns out to be too strong a statement in our setting.
%
The above shows that we have two possible ways of relaxing this, and
we have chosen the most liberal of the two options.

We also know that \(\IBoxSym\) commutes with universal quantification, i.e.\ the
formulas \(\IBox{(\IForall{y:Y}{A(y)})}\) and \(\IForall{y:Y}{\IBox{A(y)}}\) are
logically equivalent.
%
This means that, were we to place any boxes around the universal quantifiers,
we could commute them all inwards, until they would be absorbed
by our weakening of Heyting implication.

This leaves us with the final question: should we put a \(\IBoxSym\) around
the existential quantifier or not.
%
Unlike boxing the universal quantifiers, this would give us a genuinely weaker
logical statement, so it could be worthwhile.
%
However, it arguable weakens the statement too far, as it means that
the code \(e\) is no longer uniform in worlds when working in a
category of presheaves, at which point it seems we could just sheafify
our pca and work internally to the relevant category of sheaves.


\subsection{Totality}

When defining assemblies we will be particularly interested in total
predicates.

\begin{defn}\label{defn:tripos-total-predicates}
  A realizability predicate \(\varphi : \RealPred{X}\) is \definiendum{total}
  if it satisfies \(\IForall{x:X}{\IExists{a:\A}{\varphi(x,a)}}\).
\end{defn}

Building new total predicates is not particularly difficult as shown by
the following lemmas.
%
These will be enough to show that most standard constructions in the
category of assemblies have total predicates as needed.

\begin{lemm}\label{lemm:pullback-preserves-total}
  Given \(\varphi : \RealPred{Y}\) and a map \(f : X \to Y\), if
  \(\varphi\) is total then so is \(\TPull{f}\varphi\).
\end{lemm}
\begin{proof}
  Fix \(x:X\), by totality of \(\varphi\) we have some \(a:\A\) such that
  \(\varphi(f(x),a)\).
  %
  By definition, this is the same as saying \((\TPull{f}\varphi)(x,a)\) so
  pullback preserves totality.
\end{proof}

\begin{lemm}\label{lemm:top-total}
  The predicate \(\TTop{X}\) is always total.
\end{lemm}
\begin{proof}
  This is equivalent to showing that \(\A\) is inhabited, but a pca
  comes equipped with two distinguished elements so this is always the case.
\end{proof}

\begin{lemm}\label{lemm:meet-total}
  Two predicates \(\varphi,\psi : \RealPred{X}\) are total if and only if their
  meet \(\TAnd{\varphi}{\psi}{X}\) is total.
\end{lemm}
\begin{proof}
  For the forward direction, for a given \(x:X\) we have \(a:\A\) and \(b:\A\)
  such that \(\varphi(x,a)\) and \(\psi(x,b)\) and this means that
  \((\TAnd{\varphi}{\psi}{X})(x,\IAp{\IAp{\IPairCode}{a}}{b})\).
  %
  For the backwards direction, if we have \((\TAnd{\varphi}{\psi}{X})(x,a)\)
  then we also have \(\varphi(x,\IAp{\IFstCode}{a})\) and
  \(\psi(x,\IAp{\ISndCode}{a})\), so both \(\varphi\) and \(\psi\) are total.
\end{proof}

\begin{lemm}\label{lemm:join-total}
  Given two predicates \(\varphi,\psi : \RealPred{X}\), at least one of them is
  total if and only if their join \(\TOr{\varphi}{\psi}{X}\) is total.
\end{lemm}
\begin{proof}
  For the forward direction we suppose that \(\varphi\) is total
  and fix some \(x:X\).
  %
  By totality of \(\varphi\) we have \(a:\A\) such that
  \(\varphi(x,a)\), so then \((\TOr{\varphi}{\psi}{X})(x,\IAp{\ILeftCode}{a})\)
  which shows that \(\TOr{\varphi}{\psi}{X}\) is total.
  %
  In the case that \(\psi\) is total the proof follows analogously.
  %
  For the converse, fix \(x : X\), then by totality of the join
  of \(\varphi\) and \(\psi\) we have some \(a:\A\) such that
  \((\TOr{\varphi}{\psi}{X})(x,a)\).
  %
  By definition, this means that there exists \(b:\A\) such that either
  \(a = \IAp{\ILeftCode}{b}\) and \(\varphi(x,b)\), in which case \(\varphi\) is
  total, or \(a = \IAp{\IRightCode}{b}\) and \(\psi(x,b)\), in which case
  \(\psi\) is total.
\end{proof}

\begin{lemm}\label{lemm:bot-total}
  The predicate \(\TBot{X}\) is total if and only if
  \(\INot{(\IExists{x:X}\ITop)}\) holds, or equivalently,
  if \(X\) is an initial object of \(\E\).
\end{lemm}
\begin{proof}
  First we prove that totality of \(\TBot{X}\) is equivalent to
  \(\INot{(\IExists{x:X}\ITop)}\).
  %
  To do so, notice that
  \(\IForall{x:X}{\IExists{a:\A}{\TBot{X}(x,a)}}\)
  unfolds to
  \(\IForall{x:X}{\IExists{a:\A}{\IBot}}\),
  and that is equivalent to
  \(\IForall{x:X}{\IBot}\).
  %
  In turn, this is equivalent to the necessary condition.

  For the second equivalence, that \(\INot{(\IExists{x:X}\ITop)}\)
  holds if and only if \(X\) is initial, we use the Kripke-Joyal semantics
  in \(\E\).
  %
  Doing so, we see the foregoing statement holds if and only if whenever we have
  a span \(X \overset{x}{\leftarrow} Z \overset{p}{\twoheadrightarrow} Y\) with
  \(p\) an epimorphism, then \(Y\) is itself initial.
  %
  For the forward direction we have the span
  \(X \leftarrow X \twoheadrightarrow X\) where all maps are the identity,
  so by assumption \(X\) is initial.
  %
  For the converse, assume \(X\) is initial and we have a span
  \(X \overset{x}{\leftarrow} Z \overset{p}{\twoheadrightarrow} Y\).
  %
  Initial objects in a topos are strict, hence \(Z\) is also a strict initial
  object.
  %
  Any map out of a strict initial object is monic, so \(p\) is both
  monic and epic, which in a topos implies it is an isomorphism.
  %
  As \(Y\) is isomorphic to an initial object, it must also be initial
  as needed.
\end{proof}

\subsection{Presheaves with a Lawvere-Tierney Topology}

\todo[inline]{adapt to use new realizability predicate type}

Fix two realizability predicates \(\phi,\psi : \RealPred{\X}\), that is global
elements \(\phi,\psi : 1 \to \RealPred{\X}\).
%
We will unfold the denotation of \(\phi \leq \psi\) in the case of a preasheaf
category with some Lawvere-Tierney topology \(j : \Omega \to \Omega\).
%
We are, of course, interested in the following mono
%
\[
  \llbracket
    \IExists{e : \A}{
      \IForall{x : \X}{
        \IForall{a \in \varphi(x)}{
          \IBox{(
            \IAnd%
            {\ITotal{\IAp{e}{a}}}%
            {\IAp{e}{a} \in \psi(x)}
          )}
        }
      }
    }
  \rrbracket : \cdot \hookrightarrow 1
\]
%
We interpret the existential along the projection \(\pi_{\A} : 1 \times \A \to 1\)
giving us
%
\[
  \exists_{\pi_{\A}}(\underbrace{\llbracket
    \IForall{x : \X}{
      \IForall{a \in \varphi(x)}{
        \IBox{(
          \IAnd%
          {\ITotal{\IAp{e}{a}}}%
          {\IAp{e}{a} \in \psi(x)}
        )}
      }
    }
  \rrbracket}_{\text{subobject of }1 \times A})
\]
%
We interpret the universal along the projection
\(\pi_{\X} : (1 \times \A) \times \X \to 1 \times \A\) giving
%
\[
  \exists_{\pi_{\A}}\circ\forall_{\pi_{\X}}(\underbrace{\llbracket
    \IForall{a \in \varphi(x)}{
      \IBox{(
        \IAnd%
        {\ITotal{\IAp{e}{a}}}%
        {\IAp{e}{a} \in \psi(x)}
      )}
    }
  \rrbracket}_{\text{subobject of }(1 \times A) \times \X})
\]
%
Next we need to interpret the relative universal quantifier.
%
For this, the denotation of the open term
\(e : \A, x : \X \vdash \phi(x) : \IPower{\A}\) gives us a mono
%
\(
  m :
  S \hookrightarrow ((1 \times \A) \times \X) \times \A
\)
%
and composing this with the projection
\(\pi'_{\A} : ((1 \times \A) \times \X) \times \A \to (1 \times \A) \times \X\)
will give us the map along which we will interpret the quantifier.
%
\[
  \exists_{\pi_{\A}}\circ
  \forall_{\pi_{\X}}\circ
  \forall_{\pi'_{\A}\circ m}(\underbrace{\llbracket
    \IBox{(
      \IAnd%
      {\ITotal{\IAp{e}{a}}}%
      {\IAp{e}{a} \in \psi(x)}
    )}
  \rrbracket}_{\text{subobject of } S})
\]
%
Quantifiers are functorial so this is equivalent to the following
statement with a single universal
%
\[
  \exists_{\pi_{\A}}\circ
  \forall_{\pi_{\X}\circ\pi'_{\A}\circ m}(\underbrace{\llbracket
    \IBox{(
      \IAnd%
      {\ITotal{\IAp{e}{a}}}%
      {\IAp{e}{a} \in \psi(x)}
    )}
  \rrbracket}_{\text{subobject of } S})
\]
%
The modality is interpeted by taking the closure of the subobjects arising
out of the Lawvere-Tierney topology \(j:\Omega\to\Omega\).
%
\[
  \exists_{\pi_{\A}}\circ
  \forall_{\pi_{\X}}\circ
  \forall_{\pi'_{\A}\circ m}
  (\overline{\underbrace{\llbracket
      \IAnd%
      {\ITotal{\IAp{e}{a}}}%
      {\IAp{e}{a} \in \psi(x)}
  \rrbracket}_{\text{subobject of } S}})
\]
%
Conjunction is interpreted by the meet of subobjects. Such meets are computed
by pointwise intersections, but we can safely ignore that for now.
%
\[
  \exists_{\pi_{\A}}\circ
  \forall_{\pi_{\X}\circ\pi'_{\A}\circ m}
  \underset{\text{both subobjects of } S}%
  {(\overline{\underbrace{\llbracket \ITotal{\IAp{e}{a}}\rrbracket}
   \wedge
   \underbrace{\llbracket\IAp{e}{a} \in \psi(x)\rrbracket}
  })}
\]
%
At this point let us see how to interpret the particular terms mentioned.
%
We interpret the term \(e : \A, x : \X \vdash \phi(x) : \IPower{\A} \) as a
morphism \((1 \times \A) \times \X \to \IPower{\A}\) given by preocomposing
\(\EEval : X \times \RealPred{\X}\) with the pairing of the projection
\((1 \times \A) \times \X \to \X\) and of the map
\((1 \times \A) \times \X \to \RealPred{\X}\) which we get by weakening
\(\phi : 1 \to \RealPred{\X}\).
%
For clarity, the natural transformation \(\EEval\) is given by
%
\begin{align*}
  &\EEval_{w} : \X_{w} \times \mathsf{Nat}(\yo(w) \times \X, \IPower{\A})
                \to \IPower{\A}_{w}\\
  &\EEval_{w}(x, \alpha) = \alpha_{w}(\mathsf{id}, x)
\end{align*}
%
meaning that the denotation \(\llbracket\phi(x)\rrbracket\) is given by
%
\begin{align*}
  &\llbracket\phi(x)\rrbracket_{w} : (1 \times \A) \times \X \to \IPower{\A}\\
  &\llbracket\phi(x)\rrbracket_{w}((\star, a), x) =
    \EEval_{w}(x, \phi_{w}) = \phi_{w}(\mathsf{id}, x)
\end{align*}
%
Naturality in this case means that for any \(f : u \to w\), \(g : v \to u\)
and \(x : \X_{u}\) we have
\[
  (\alpha_{u}(f, x))\{g\} = \alpha_{v}(fg, x\{g\})
\]


\[
  ([\alpha_{v}(fg, x\{g\})]_{u}(f, a)) = \alpha_{v}(fg, x\{g\})
\]

We can now define the mono \(m\) from \(\llbracket\phi(x)\rrbracket\) as
the mono corresponding to the characteristic function
\(\chi_{m} : (1 \times \A) \times \X \to \Omega\) arising by precomposing
with the interpretations of \(e : \A, x : \X \vdash a : \A  \)
and the weakening of \(e : \A, x : \X \vdash \phi(x) : \IPower{\A}\).
%
In the end this means that we have
\[
  (((\star, e), x), a) \in \mathsf{domain}(m)_{w}
  \iff
  w \in [\phi_{w}(\mathsf{id}, x)]_{w}(\mathsf{id}, a)
\]

\dots

And this leads us to the following external definition of the ordering
on realizability predicates.

\begin{defn}
  We say that \(\phi \leq \psi\) at world \(w\) if there exists a code
  \(e : \A_{w}\) such that for all maps \(f : u \to w\), elements
  \(x : \X_{u}\) and codes \(a : \A_{u}\), if
  %
  \[
    \mathsf{id}_{u} \in [\phi_{u}(\mathsf{id}_{u}, x)]_{u}(\mathsf{id}_{u},a)
  \]
  %
  then there exists a cover \(\mathcal{V}\) of \(u\) such that for all
  \((g : v \to u) \in \mathcal{V}\) we have
  %
  \[\begin{array}{ccc}
  e\{fg\} \cdot_{v} a\{g\} \downarrow &\text{ and }&
  \mathsf{id}_{v} \in [\psi_{v}(\mathsf{id}_{v}, x\{g\})]_{v}(\mathsf{id}_{v},e\{fg\} \cdot_{v} a\{g\}).
  \end{array}\]
\end{defn}

In the case that the underlying category is a poset
%
\begin{defn}[Over a poset]
  We say that \(\phi \leq \psi\) at world \(w\) if there exists a code
  \(e : \A_{w}\) such that for all extensions \(u \leq w\), elements
  \(x : \X_{u}\) and codes \(a : \A_{u}\), if
  %
  \[
    u \in [\psi_{u}(u\leq u, x)]_{u}(u \leq u,a)
  \]
  %
  then there exists a cover \(\mathcal{V}\) of \(u\) such that for all \(v \in \mathcal{V}\) we
  have
  %
  \[\begin{array}{ccc}
    e|_{v} \cdot_{v} a|_{v} \downarrow &\text{ and }&
    v \in [\psi_{v}(v\leq v, x|_{v})]_{v}(v \leq v,e|_{v} \cdot_{v} a|_{v})
  \end{array}\]
\end{defn}

And further assuming that \(\A\) is a constant presheaf we get
%
\begin{defn}[Over a poset with constant \(\A\)]
  We say that \(\phi \leq \psi\) at world \(w\) if there exists a code
  \(e : \A\) such that for all extensions \(u \leq w\), elements
  \(x : \X_{u}\) and codes \(a : \A\), if
  %
  \[
    u \in [\phi_{u}(u\leq u, x)]_{u}(u \leq u,a)
  \]
  %
  then there exists a cover \(\mathcal{V}\) of \(u\) such that for all \(v \in \mathcal{V}\) we
  have
  %
  \[\begin{array}{ccc}
    e \cdot_{v} a \downarrow &\text{ and }&
    v \in [\psi_{v}(v\leq v, x|_{v})]_{v}(v \leq v,e \cdot_{v} a)
  \end{array}\]
\end{defn}

We can also work with uncurried predicates, so now we have
\(\phi,\psi : \IPower{(\X\times\A)}\), the previous assumptions and if we
also omit the proofs that specific worlds are extensions of others, then
we get the following definition, which is what we have been working with
already:
%
\begin{defn}[Lots of simplifications]
  We say that \(\phi \leq \psi\) at world \(w\) if there exists a code
  \(e : \A\) such that for all extensions \(u \leq w\), elements
  \(x : \X_{u}\) and codes \(a : \A\), if
  %
  \[
    u \in \phi_{u}(x, a)
  \]
  %
  then there exists a cover \(\mathcal{V}\) of \(u\) such that for all \(v \in \mathcal{V}\) we
  have
  %
  \[\begin{array}{ccc}
    e \cdot_{v} a \downarrow &\text{ and }&
    v \in \psi_{v}(x|_{v}, e \cdot_{v} a)
  \end{array}\]
\end{defn}

\section{Realizability Categories}%
\label{sec:realizability-categories}

With a tripos we can now define its associated category of assemblies
and realizability topos.

\subsection{Assemblies}%
\label{sub:assemblies}

\begin{defn}\label{defn:assembly}
  An \definiendum{assembly} is an object \(X\) of \(\E\) with a realizability
  predicate \(\varphi : \RealPred{X}\) satisfying the following internal
  statement \(\IForall{x:X}{\IExists{a:A}{\varphi(x,a)}}\).
  %
  We call such a predicates \definiendum{total}.
\end{defn}

\begin{defn}\label{defn:assembly-morphism}
  A \definiendum{morphism of assemblies} from \((X,\varphi)\) to
  \((Y,\psi)\) consists of a morphism \(f : X \to Y\) in \(\E\) such that
  \(\varphi \leq \TPull{f}\psi\). When \(\varphi \leq \TPull{f}\psi\) holds
  we say that \(f\) is \definiendum{tracked}.
\end{defn}

\begin{remk}
  Using the internal language of \(\E\), \(f : X \to Y\) is tracked as above
  if we have
  %
  \[
    \IExists{e : \A}{
      \IForall{x : \X}{
        \IForall{a : \A}{
          \IImplies%
            {\varphi(a,x)}%
            {\IBox{(
              \IAnd%
              {\ITotal{\IAp{e}{a}}}%
              {\psi(\IAp{e}{a}, f(x))}
            )}}
        }
      }
    }
  \]
\end{remk}

Though we will sometimes use this internal definition, other times it will be
easier to use abstract properties of triposes, for example in showing that
assemblies form a category.

\begin{cons}
  The \definiendum{category of assemblies}, denoted \(\AsmCat\), has as objects
  the assemblies and as morphisms the morphisms of assemblies.
  %
  Identity morphisms and composition are both inherited from \(\E\).
\end{cons}
\begin{proof}
  We claimed that identity morphisms and composition are inherited from \(\E\).
  %
  For the former, given an assembly \((X,\varphi)\) we have the identity
  \(\IId{X}\) which is tracked since \(\TPull{\IId{X}}\) is the identity and
  \(\leq\) is reflexive.
  %
  For composition, suppose we have morphisms of assemblies
  \(f : (Y,\psi) \to (X,\varphi)\) and \(g : (Z,\chi)\to(Y,\psi)\).
  %
  Both of these are tracked hence so \(\psi \leq \TPull{f}\varphi\) and
  \(\chi \leq \TPull{g}\psi\), but \(\TPull{g}\) is monotonic so in addition we
  also have
  \(\chi \leq \TPull{g}\psi \leq \TPull{g}\TPull{f}\varphi=\TPull{(fg)}\varphi\)
  which means that \(fg\) is tracked as necessary.
  %
  The unit and associativity equalities then follow immediately as \(\E\) is a
  category.
\end{proof}

\begin{lemm}\label{lemm:assemblies-finite-limits}
  The category of assemblies \(\AsmCat\) has all finite limits.
\end{lemm}
\begin{proof}
  We construct binary products and equalisers from which all finite limits
  follow.

  Fix two assemblies \((X,\varphi)\) and \((Y,\psi)\), the underlying object of
  their product will be the product \(\IProd{X}{Y}\) and the realizability
  relation will be given by
  \(\TAnd{(\TPull{\IFst}\varphi)}{(\TPull{\ISnd}\psi)}{\IProd{X}{Y}}\).
  %
  This is total by lemmas~\ref{lemm:pullback-preserves-total}
  and~\ref{lemm:meet-total}.
  %
  The projections \(\IFst\) and \(\ISnd\) are tracked as we have
  \(\TAnd{(\TPull{\IFst}\varphi)}{(\TPull{\ISnd}\psi)}{\IProd{X}{Y}} \leq \TPull{\IFst}\varphi\)
  and
  \(\TAnd{(\TPull{\IFst}\varphi)}{(\TPull{\ISnd}\psi)}{\IProd{X}{Y}} \leq \TPull{\ISnd}\psi\).
  %
  This leaves us to show that given morphisms \(f : (Z,\chi) \to (X,\varphi)\)
  and \(g : (Z,\chi) \to (Y,\psi)\) that \(\IPair{f}{g} : Z \to \IProd{X}{Y}\)
  is tracked.
  %
  Since \(f\) and \(g\) are tracked we know that
  \[
    \chi \leq \TAnd{(\TPull{f}\varphi)}{(\TPull{g}\psi)}{Z}
         = \TAnd{\TPull{(\IFst\IPair{f}{g})}\varphi}{\TPull{(\ISnd\IPair{f}{g})}\psi}{Z}
         = \TPull{\IPair{f}{g}}\left[\TAnd{(\TPull{\IFst}\varphi)}{(\TPull{\ISnd}\psi)}{\IProd{X}{Y}}\right]
  \]
  which shows that \(\IPair{f}{g}\) is tracked and so the category of assemblies
  has binary products.

  For equalisers fix two morphisms \(f,g : (X,\varphi) \to (Y,\psi)\) and
  denote their equaliser in \(\E\) by \(m : \IEqualiser{f}{g} \to X\).
  %
  We turn this equaliser into an assembly by giving it the realizability
  predicate \(\TPull{m}\varphi\), the totality of which follows from lemma
  \ref{lemm:pullback-preserves-total}.
  %
  To show \(m\) is tracked amounts to proving
  \(\TPull{m}\varphi \leq \TPull{m}\varphi\), which holds by reflexivity of
  \(\leq\).
  %
  Finally suppose we have \(h : (Z,\chi) \to (X,\varphi)\) which equalises \(f\)
  and \(g\), then the universal property of \(\IEqualiser{f}{g}\) tells us there
  exists a unique map \(u : Z \to \IEqualiser{f}{g}\) such that \(h = mu\).
  %
  As \(h\) is tracked, then so is \(u\) which follows from
  \(\chi\leq\TPull{h}\varphi=\TPull{(mu)}\varphi=\TPull{u}(\TPull{m}\varphi)\).
\end{proof}

\begin{lemm}\label{lemm:assemblies-cartesian-closed}
  The category of assemblies \(\AsmCat\) is cartesian closed.
\end{lemm}
\begin{proof}
  From \Cref{lemm:assemblies-finite-limits} we know that binary products exist,
  it thus remains to show that exponentials exist.
  %

\end{proof}

\subsection{Realizability Topos}%
\label{sub:realizability-topos}

The definition of the realizability topos follows by simply
unfolding the tripos-to-topos construction so we don't focus on it for now.
%
Eventually it would be good to look at assemblies as a full subcategory of the
realizability topos given by the subobjects of the constant objects, and
compare these to the more elementary, direct definition of assemblies.

\subsection{Sheafififcation}

Usually sheaves are described as a reflective subcategory of a category
of presheaves.
%
That means we have an adjoint pair of functors \(L \dashv R\) with
\(R\) fully faithful.
%
The domain of \(R\) is then the category of sheaves, and its codomain the
category of presheaves.
%
We view \(R\) as the inclusion of sheaves into presheaves and \(L\)
as the sheafification of presheaves.
%
From the adjointness condition we recover the property that mapping out
of the sheafification of a presheaf into a sheaf is equivalent to mapping out
of the original presheaf into the same sheaf.

For now we look for an analogous sheafification operation in the category
of assemblies.
%
Presumably, the action of said operation on the underlying presheaves
will be actual sheafification.
%
That leaves the action on the realizability relations up for debate.

For our first attempt, let us deconstruct this in the same way that
sheafification is deconstructed, so we have a single operation
\({(-)}^{+}\) which sends an arbitrary presheaf to a separated presheaf,
and a separated presheaf to a sheaf.
%
Then sheafification can be done by iterating the \(+\)-operation twice.

\todo[inline]{develop this more}


\newpage
\appendix

\section{Some intuitions from Set-based realizability}

Let us focus on standard set-based realizability for now, where the connection
between assemblies and the realizability topos is well-understood.
%
Given a set-based tripos \(P\), its realizability topos is given by the
following:

\begin{defn}\label{defn-appendix-realizability-topos}
  The \definiendum{objects of the realizability topos} consist of a set \(X\)
  along with a binary relation \({\approx} : P(\IProd{X}{X})\) which is
  \[\begin{array}{ll}
    \textbf{(symmetric)}& x:X,\,y:X \mid x \approx y \vdash y \approx x\\
    \textbf{(transitive)}& x:X,\,y:X,\,z:X \mid x \approx y,\,y \approx z \vdash x \approx z
  \end{array}\]
  %
  The \definiendum{morphisms of the realizability topos}
  from \((X,{ \approx_{X} })\) to \((Y,{ \approx_{Y} })\) consist of
  equivalence classes of binary relations \(F : P(\IProd{X}{Y})\) which are
  \[\begin{array}{ll}
      \textbf{(strict)}& x:X,\,y:Y \mid
                         F(x,y) \vdash
                         x\approx_{X}x \land y\approx_{Y}y\\
      \textbf{(total)}& x:X, \mid
                         x\approx_{X}x \vdash
                         \exists y:Y, F(x,y)\\
      \textbf{(extensional)}&
                              x:X,\,x':X,\,y:Y,\,y':Y \mid
                              x\approx_{X}x',\,F(x,y)\,F(x',y') \vdash
                              y\approx_{Y}y'
  \end{array}\]
  where two such relations \(F,G\) are in the same equivalence class if
  \(x:X,\,y:Y\mid F(x,y) \dashv\vdash G(x,y)\).
\end{defn}

So although morphisms are given by relations, the extra conditions imposed on
them mean they behave a lot more like functions than maybe is apparent at first.
%
More specifically, given an object \(X,\approx_{X}\), we should think of it as
the set \(\{x : X \mid x\approx_{X}x\}\) quotiented by \(\approx_{X}\).
%
Under this view, a morphism from \((X,\approx_X)\) to \((Y,\approx_Y)\)
does give an actual function.
%
Of course, this is mainly an intuition: the objects and morphisms actually
contain some ``computability'' information from the tripos \(P\) which is lost
this way.

%\todo[inline]{following paragraph is wrong, can I fix it?}
%Expanding on this previous point, let us now consider the effective topos, that
%is the realizability topos associated with Kleene's first algebra
%\(\mathcal{K}_{1}\).
%%
%In the effective topos we have the
%objects \((\mathbb{N}, \sim)\) and \((\mathbb{N}, \approx)\),
%both with underlying sets the natural numbers and the following
%relations:
%%
%\[
%  \begin{array}{cc}
%    n \sim m \IsDefined
%      \begin{cases}
%        \{n\} & \text{ if } n = m \\
%        \emptyset & \text{ otherwise }
%      \end{cases}
%    &\qquad\qquad n \approx m \IsDefined \{0\}
%  \end{array}
%\]
%%
%The first of these objects is quite canonical and in fact turns out to be the
%natural number object in the effective topos.
%%
%The latter is a little less canonical
%be just enough to get some weird results.
%%
%Regardless, they are both symmetric and transitive: in the case of \(\sim\)
%these are both realized by the identity function and in the case of \(\approx\)
%they are both realized by the constantly zero function.
%%
%We have the following two relations \(F\) and \(G\) that represent morphisms
%from \((\mathbb{N},\sim)\) to \((\mathbb{N},\approx)\):
%%
%\[
%  \begin{array}{cc}
%    F(n, m) \IsDefined
%      \begin{cases}
%        \{n\} & \text{ if } n = m \\
%        \emptyset & \text{ otherwise }
%      \end{cases}
%    &\qquad\qquad G(n, m) \IsDefined
%      \begin{cases}
%        \{0\} & \text{ if } n = m \\
%        \emptyset & \text{ otherwise }
%      \end{cases}
%  \end{array}
%\]
%%
%These relations are indeed both strict, total and extensional.
%%
%In proving strictness of \(F\) we require

\printbibliography{}

\end{document}
