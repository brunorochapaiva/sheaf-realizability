\documentclass[11pt]{article}


\newif\ifnotes
\notestrue % comment out to hide notes

\DeclareUnicodeCharacter{03A3}{\(\Sigma\)}
\DeclareUnicodeCharacter{03A0}{\(\Pi\)}

\usepackage{geometry}
\geometry{
  a4paper,
  left=25mm,
  right=25mm,
  top=25mm,
}

% Packages for using unicode
\usepackage[utf8]{inputenc}
\usepackage[english]{babel}

\usepackage[T1]{fontenc}
%\usepackage[scaled=0.75]{DejaVuSansMono} % Good monospace font for code

\usepackage[
  backend=biber,
  mincrossrefs=999,
  style=numeric,
  doi=true,
  isbn=false,
  url=false,
  eprint=false,
]{biblatex}
\addbibresource{references.bib}
\AtEveryBibitem{\clearname{editor}} % Always remove editor information

\usepackage[]{hyperref}
\hypersetup{
    colorlinks=false,
}

\usepackage{amsmath}
\usepackage{amssymb}
\usepackage{amsfonts}
\usepackage{amsthm}
\usepackage{newtxmath}
\usepackage{url}
\usepackage{tikz-cd}
\usepackage{quiver}
\usepackage{scalefnt}
\usepackage{microtype}
\usepackage{subfiles}
\usepackage{mathpartir}
\usepackage[bb=dsserif]{mathalpha}
\usepackage{bm}
\usepackage{mathtools}
\ifnotes%
\usepackage{todonotes}
\else
\usepackage[disable]{todonotes}
\fi
\usepackage{cleveref}
\usepackage{xspace}
\usepackage{multirow}
\usepackage{ulem}
\usepackage{contour}


\newtheorem{thrm}{Theorem}[section]
\newtheorem{lemm}[thrm]{Lemma}
\newtheorem{prop}[thrm]{Proposition}
\newtheorem{defn}[thrm]{Definition}
\newtheorem{remk}[thrm]{Remark}
\newtheorem{exam}[thrm]{Example}
\newtheorem{cons}[thrm]{Construction}
\newtheorem{coro}[thrm]{Corollary}
\newtheorem{nota}[thrm]{Notation}

%% Macros

%% Macro to superimpose two symbols
\makeatletter
\newcommand{\superimpose}[3][\mathord]{#1{\mathpalette\superimpose@{{#2}{#3}}}}
\newcommand{\superimpose@}[2]{\superimpose@@{#1}#2}
\newcommand{\superimpose@@}[3]{%
  \ooalign{%
    \hfil$\m@th#1#2$\hfil\cr
    \hfil$\m@th#1#3$\hfil\cr
  }%
}
\makeatother

%%%%%%%%%%%%%%%%%%%%%%%%%%%%%%%%%%%%%%%%%%%%%%%%%%%%%%%%%%%%%%%%%%%%%%%%%%%%%%%%
%% GENERAL MATHEMATICAL NOTATION %%%%%%%%%%%%%%%%%%%%%%%%%%%%%%%%%%%%%%%%%%%%%%%
%%%%%%%%%%%%%%%%%%%%%%%%%%%%%%%%%%%%%%%%%%%%%%%%%%%%%%%%%%%%%%%%%%%%%%%%%%%%%%%%

%% Used in the vernacular to mark a term being defined.
%% Currently, does nothing else besides making the term italic.
%% But it’s still nice to provide a layer of abstraction for this purpose.
%% It may be used, for example, to add things to the index.
\newcommand{\definiendum}[1]{\textbf{#1}}

\newcommand{\IsDefined}{\mathbin{\vcentcolon\equiv}}

%% General one letter notations
\newcommand{\A}{\ensuremath{\mathsf{A}}} % arbitrary PCA
\newcommand{\C}{\ensuremath{\mathcal{C}}} % arbitrary category
\renewcommand{\P}{\ensuremath{{P}}} % arbitrary poset (replaces paragraph symbol)
\renewcommand{\L}{\ensuremath{{L}}} % arbitrary lattice
\newcommand{\X}{\ensuremath{X}} % arbitrary sheaf
\newcommand{\E}{\mathcal{E}} % arbitrary topos
\newcommand{\Ej}{\mathcal{E}_\IBoxSym} % arbitrary topos of sheaves

\newcommand{\Par}[1]{{#1}_\bot}
\newcommand{\ParTo}{\rightharpoonup}
\newcommand{\ParCat}[1]{{#1}^\ParTo}

\newcommand{\PresheafCat}{\hat{\C}}

\newcommand{\Sub}[1]{\mathsf{Sub}({#1})}

%%%%%%%%%%%%%%%%%%%%%%%%%%%%%%%%%%%%%%%%%%%%%%%%%%%%%%%%%%%%%%%%%%%%%%%%%%%%%%%%
%% VERNACULAR COMMANDS %%%%%%%%%%%%%%%%%%%%%%%%%%%%%%%%%%%%%%%%%%%%%%%%%%%%%%%%%
%%%%%%%%%%%%%%%%%%%%%%%%%%%%%%%%%%%%%%%%%%%%%%%%%%%%%%%%%%%%%%%%%%%%%%%%%%%%%%%%

\newcommand{\SystemT}{\textsc{System T}}
\newcommand{\MLTT}{\textsc{MLTT}}
\newcommand{\Agda}{\textsc{Agda}}
\newcommand{\Boxtt}{\textsc{BoxTT}}

%%%%%%%%%%%%%%%%%%%%%%%%%%%%%%%%%%%%%%%%%%%%%%%%%%%%%%%%%%%%%%%%%%%%%%%%%%%%%%%%
%% DECORATION %%%%%%%%%%%%%%%%%%%%%%%%%%%%%%%%%%%%%%%%%%%%%%%%%%%%%%%%%%%%%%%%%%
%%%%%%%%%%%%%%%%%%%%%%%%%%%%%%%%%%%%%%%%%%%%%%%%%%%%%%%%%%%%%%%%%%%%%%%%%%%%%%%%

\setlength{\ULdepth}{3pt}
\contourlength{1pt}
\newcommand{\underlinehelper}[3]{
  \colorbox{#2}{
  #1{\phantom{#3}}%
  \llap{\contour{#2}{#3}}%
  }
}
\newcommand{\hla}[1]{\underlinehelper{\uline}{pink}{#1}}
\newcommand{\hlb}[1]{\underlinehelper{\uwave}{yellow}{#1}}
\newcommand{\hlc}[1]{\underlinehelper{\uuline}{pink}{#1}}

%%%%%%%%%%%%%%%%%%%%%%%%%%%%%%%%%%%%%%%%%%%%%%%%%%%%%%%%%%%%%%%%%%%%%%%%%%%%%%%%
%% EXTERNAL %%%%%%%%%%%%%%%%%%%%%%%%%%%%%%%%%%%%%%%%%%%%%%%%%%%%%%%%%%%%%%%%%%%%
%%%%%%%%%%%%%%%%%%%%%%%%%%%%%%%%%%%%%%%%%%%%%%%%%%%%%%%%%%%%%%%%%%%%%%%%%%%%%%%%

%% Eval map
\newcommand{\EEval}{\mathsf{ev}}

%%%%%%%%%%%%%%%%%%%%%%%%%%%%%%%%%%%%%%%%%%%%%%%%%%%%%%%%%%%%%%%%%%%%%%%%%%%%%%%%
%% INTERNAL %%%%%%%%%%%%%%%%%%%%%%%%%%%%%%%%%%%%%%%%%%%%%%%%%%%%%%%%%%%%%%%%%%%%
%%%%%%%%%%%%%%%%%%%%%%%%%%%%%%%%%%%%%%%%%%%%%%%%%%%%%%%%%%%%%%%%%%%%%%%%%%%%%%%%

% Logical language

\newcommand{\ITop}{\top}

\newcommand{\IBot}{\bot}

\newcommand{\INotSym}{\lnot}
\newcommand{\INot}[1]{\INotSym{#1}}

\newcommand{\IAndSym}{\land}
\newcommand{\IAnd}[2]{{#1}\IAndSym{#2}}

\newcommand{\IOrSym}{\lor}
\newcommand{\IOr}[2]{{#1}\IOrSym{#2}}

\newcommand{\IImpliesSym}{\Rightarrow}
\newcommand{\IImplies}[2]{{#1}\IImpliesSym{#2}}

\newcommand{\IForallSym}{\forall}
\newcommand{\IForall}[2]{\IForallSym{#1}.\,{#2}}

\newcommand{\IExistsSym}{\exists}
\newcommand{\IExists}[2]{\IExistsSym{#1}.\,{#2}}

\newcommand{\IBoxSym}{\Box}
\newcommand{\IBox}[1]{\IBoxSym{#1}}

\newcommand{\ISetComp}[2]{\{{#1}\mid{#2}\}}

\newcommand{\IEqSym}{=}
\newcommand{\IEq}[2]{{#1}\IEqSym{#2}}

% Types
\newcommand{\IPower}[1]{\mathcal{P}{#1}}
\newcommand{\IPowerCl}[1]{\mathcal{P}_{\IBoxSym}{#1}}
\newcommand{\IProp}{\Omega}
\newcommand{\IPropCl}{\Omega_\IBoxSym}
\newcommand{\IHolds}[1]{|{#1}|}

%% Universe
\newcommand{\IUni}{\mathcal{U}}

%% Empty type
\newcommand{\IEmpty}{\mathbb{0}}
\newcommand{\IEmptyM}[1]{!_{#1}}

%% Unit type
\newcommand{\IUnit}{\mathbb{1}}
\newcommand{\IUnitM}[1]{{\langle\rangle}_{#1}}

%% Product types
\newcommand{\IProd}[2]{{#1}\times{#2}}
\newcommand{\IFst}{\pi_0}
\newcommand{\ISnd}{\pi_1}
\newcommand{\IPair}[2]{\langle{#1},{#2}\rangle} % pairing operator
\newcommand{\IPairBi}[2]{{#1}\times{#2}} % bifunctor action of products

%% Equaliser types
\newcommand{\IEqualiser}[2]{\mathsf{Eq({#1,#2})}}

%% Function types
\newcommand{\IFun}[2]{{{#1}\to{#2}}}
\newcommand{\ILam}[2]{{\mathop{\lambda({#1}).}{#2}}}

\newcommand{\IId}[1]{\mathbb{1}_{#1}}

% INTERNAL PCA

%% Partial application
\newcommand{\IApSym}{\cdot}
\newcommand{\IAp}[2]{{#1}\IApSym{#2}}

%% Polynomial application
\newcommand{\IPApSym}{\cdot}
\newcommand{\IPAp}[2]{{#1}\IPApSym{#2}}

%% Polynomial abstraction
\newcommand{\IAbs}[2]{\Lambda{#1}.\,{#2}}

%% Partial element ordering
\newcommand{\IPLeqSym}{\preccurlyeq}
\newcommand{\IPLeq}[2]{{#1}\IPLeqSym{#2}}

\newcommand{\IPSimSym}{\simeq}
\newcommand{\IPSim}[2]{{#1}\IPSimSym{#2}}

\newcommand{\IIdentCode}{\texttt{id}}

\newcommand{\ICompCode}{\texttt{comp}}

%% Pairing
\newcommand{\IPairCode}{\texttt{pair}}
\newcommand{\IFstCode}{\texttt{p}_0}
\newcommand{\ISndCode}{\texttt{p}_1}

%% Disjoint unions
\newcommand{\ILeftCode}{\texttt{in}_0}
\newcommand{\IRightCode}{\texttt{in}_1}
\newcommand{\IMatchCode}{\texttt{match}}


%% Sheafification
\newcommand{\ISheafSym}{\mathcal{D}}
\newcommand{\ISheaf}[1]{\ISheafSym{#1}}
\newcommand{\ISheafExt}[1]{{#1}^\sharp}
\newcommand{\ISheafIndSym}{\mathcal{D}_\text{ind}}
\newcommand{\ISheafInd}[1]{\ISheafIndSym\,{#1}}

\newcommand{\IGlueSym}{\text{glue}}
\newcommand{\IGlue}[1]{\IGlueSym_{#1}}

\newcommand{\ILeafSym}{\eta}
\newcommand{\ILeaf}[1]{\eta\,{#1}}

\newcommand{\IBranchSym}{\beta}
\newcommand{\IBranch}[3]{\beta(#1,#2,#3)}

%% Lifting monad

\newcommand{\IPar}[1]{{#1}_\bot}
\newcommand{\ITotal}[1]{{{#1}\!\downarrow}}
\newcommand{\IValSym}{\text{value}}
\newcommand{\IVal}[1]{\IValSym\,{#1}}

\newcommand{\ISquashSym}{{\Downarrow}}
\newcommand{\ISquash}[1]{\ISquashSym\,{#1}}

\newcommand{\IParExtSym}{\Diamond}
\newcommand{\IParExt}[1]{\IParExtSym{#1}}

%%%%%%%%%%%%%%%%%%%%%%%%%%%%%%%%%%%%%%%%%%%%%%%%%%%%%%%%%%%%%%%%%%%%%%%%%%%%%%%%
%% TRIPOS %%%%%%%%%%%%%%%%%%%%%%%%%%%%%%%%%%%%%%%%%%%%%%%%%%%%%%%%%%%%%%%%%%%%%%
%%%%%%%%%%%%%%%%%%%%%%%%%%%%%%%%%%%%%%%%%%%%%%%%%%%%%%%%%%%%%%%%%%%%%%%%%%%%%%%%

%% Tripos
\newcommand{\Trip}{T_\IBoxSym}

%% Realizability Predicates
\newcommand{\RealPred}[1]{\IPowerCl{(\IProd{#1}{\A})}}

% Tripos language

%% Helper macro to tag tripos language symbols with the object type
\newcommand{\TagWithObject}[2]{\overset{\scriptscriptstyle #2}{#1}}

\newcommand{\TTop}[1]{\TagWithObject{\top}{#1}}

\newcommand{\TBot}[1]{\TagWithObject{\bot}{#1}}

\newcommand{\TAndSym}[1]{\TagWithObject{\land}{#1}}
\newcommand{\TAnd}[3]{{#1}\TAndSym{#3}{#2}}

\newcommand{\TOrSym}[1]{\TagWithObject{\lor}{#1}}
\newcommand{\TOr}[3]{{#1}\TOrSym{#3}{#2}}

\newcommand{\TImpliesSym}[1]{\TagWithObject{\Rightarrow}{#1}}
\newcommand{\TImplies}[3]{{#1}\TImpliesSym{#3}{#2}}

\newcommand{\TPull}[1]{{#1}^\ast}
\newcommand{\TExists}[1]{\exists_{#1}}
\newcommand{\TForall}[1]{\forall_{#1}}

%% Category of assemblies
\newcommand{\AsmSym}{\mathbf{Asm}}
\newcommand{\Asm}[3]{\AsmSym_{(#2,#3)}(#1)}
\newcommand{\AsmCat}{\Asm{\A}{\E}{\Box}}

\newcommand{\Adj}[2]{{#1}[{#2}]}


%--------------------------------------------------------------------------------
% Document proper
\begin{document}

\title{Realizability with Sheaves}
\author{Bruno da Rocha Paiva}
\maketitle

We begin a treatment of realizability internal to categories of sheaves.

\newpage
\section{Internal Language of a Presheaf Topos}

A typical course in mathematical logic may start by looking at formal systems
such as natural deduction or the sequent calculus, giving rise to a notion of
truth, namely the provability of formulas. A natural next step is to then look
at the semantics of said proof system. These semantics can vary significantly
depending on the chosen logic such as boolean valuations for propositional
logic, Kripke frames for modal logic, \(\mathcal{L}\)-structures for predicate
logic, etc. Regardless, in each case we can consider whether a formula is valid,
i.e.\ true in every model and this gives rise to yet to a second notion of
truth. Remarkably, these two notions of truth often coincide, provided the right
proof system and semantics are chosen. While provability quantifies
existentially over proofs, asking whether a formal proof exists, validity is a
universal statement asking that the interpretation of a formula hold in all
models. This duality is powerful: to show a formula \(\phi\) is not provable by
proof-theoretic means is difficult. Typical methods start by identifying some
property of proofs \(P\) such that any proof of any formula \(\psi\) can be
transformed to another proof of \(\psi\) satisfying \(P\). From here, if we can
show that no proof of \(\phi\) satisfying \(P\) exists, then that must mean no
proof exists in general. Typical examples of properties \(P\) can be that the
proof does not make use of the \textbf{Cut} rule in sequent calculus or
alternatively for natural deduction that there exist no \(\beta\)-redexs. While
these are interesting and useful results to have, in this case the semantics
offers an easier route. To show \(\phi\) is not valid it suffices to demonstrate
a model in which the interpretation of \(\phi\) does not hold and this is
generally easier than the syntactic route.

At a first glance the power of this duality seems to be in dispatching of syntax
and work purely with models. As we move to more and more complicated languages,
going beyond first-order logic to its higher-order counterparts, then the needle
begins to move back. In particular, consider (your favourite version of)
Martin-L\"of Type Theory (\MLTT{}) with \(\Pi\)-types and \(\Sigma\)-types.
Semantics for \MLTT{} are usually categorically, either through categories with
families, contextual categories, natural models, the list goes on. Fixing your
favourite notion of model, categories of presheaves \(\PresheafCat\) generally
give rise to a model of dependent type theory. That means if we prove some
statement in \MLTT{}, then by taking the interpretation of said statement in
\(\PresheafCat\) we will get a true statement about presheaves over \(\C\).
Working with presheaves can be quite labourious and often becomes an exercise
in bookkeeping of indices.



\newpage
\section{Internal PCAs}

In set-based mathematics, partial functions \(f : X \ParTo Y\) are usually
defined as a subset \(\text{dom}(f) \subseteq X\) and a total function
\(f : \text{dom}(f) \to Y\) on said subset. With such a definition, the
composition of two partial functions \(f : X \ParTo Y\) and \(g : Y \ParTo Z\)
will have domain \(\text{dom}(g \circ f) = \text{im}(f) \cap \text{dom}(g)\), on
which the composite \(g \circ f\) is total. Associativity of function
composition comes down to associativity of subset intersections and as a total
function is ``trivially'' partial we also have identity maps satisfying the unit
laws. With this, we have just defined a category of sets and partial functions
between them. A similar construction can be done for arbitrary categories under
the mild condition that pullbacks along monomorphisms exist. Said pullbacks will
provide composition of partial morphisms, leading us to the final obstruction,
which is that pullbacks are only associative up to (unique) isomorphism. This is
remedied by identifying partial maps up to isomorphism of the domains as we will
make explicit later.

\begin{defn}
  Given a category \(\C\) with pullbacks we define the \definiendum{category of
    partial maps} in \(\C\), denoted \(\ParCat{\C}\), according to the following:
  \begin{itemize}
    \item objects of \(\ParCat{\C}\) are exactly those of \(\C\).
    \item given objects \(X,Y\), their hom-set \(\ParCat{\C}(X,Y)\) is given by
      equivalence classes of spans \(X \xhookleftarrow{m} S \xrightarrow{f} Y\)
      with \(m\) monic. Two such spans
      \(X \xhookleftarrow{m} S \xrightarrow{f} Y\) and
      \(X \xhookleftarrow{n} P \xrightarrow{g} Y\) are equivalent if there
      exists an isomorphism \(f : S \cong P\) such that the following commutes
      \[\begin{tikzcd}
          && S \\
          X &&&& Y \\
          && P \arrow["m"', hook, from=1-3, to=2-1] \arrow["f", from=1-3, to=2-5] \arrow["f"', from=1-3, to=3-3] \arrow["n", hook', from=3-3, to=2-1] \arrow["g"', from=3-3, to=2-5]
        \end{tikzcd}\]
    \item Composition of two morphisms
      \(X \xhookleftarrow{m} S \xrightarrow{f} Y\) and
      \(Y \xhookleftarrow{n} P \xrightarrow{g} Z\) is given by the pullback
      \[\begin{tikzcd}
          X \\
          S & Y \\
          {S\times_YP} & P & Z \arrow["m", hook', from=2-1, to=1-1] \arrow["f"', from=2-1, to=2-2] \arrow[from=3-1, to=2-1] \arrow["\lrcorner"{anchor=center, pos=0.125, rotate=90}, draw=none, from=3-1, to=2-2] \arrow[from=3-1, to=3-2] \arrow["n", hook', from=3-2, to=2-2] \arrow["g"', from=3-2, to=3-3]
        \end{tikzcd}\]
    \item Identity maps are given by the span consisting of two identity maps.
  \end{itemize}
\end{defn}

The associative and unit laws can be checked to follow from the uniqueness (up
to isomorphism) of pullbacks. Working through this definition for the category
of sets we recover a category isomorphic to our original description. In the
specific case of sets we were able to avoid worrying about equivalence classes
by using subsets instead of subobjects and subset intersections instead of
pullbacks. Similarly, we can give a more concrete description for the category
of partial maps of sheaves.

\newpage

Fix an internal PCA \(\A\) in the category of presheaves on \(\C\) along with a
coverage yielding the modality \(\Box\). We order predicates using the
following internal statement
\[
  \varphi \leq \psi
  \iff
  \PresheafCat \models
    \exists e : A,
    \forall x : X,
    \forall a \in \varphi(x),
    \Box ({e \odot a \downarrow} \land e \odot a \in \psi(x))
\]
Let us now compile this out to the corresponding external statement about
\(\PresheafCat\). We will start assuming nothing about \(\A\) or
\(\C\) to get the general definition. Then, we will look at the specific
case when \(\C\) is a pre-order and \(\A\) is a constant presheaf to compare
to previous work.

The interpretation of the foregoing statement will be a subpresheaf of
the terminal presheaf, which is constantly singleton. That is,
we have
\[
 \llbracket
    \exists e : \A,
    \forall x : X,
    \forall a \in \varphi(x),
    \Box ({e \odot a \downarrow} \land e \odot a \in \psi(x))
  \rrbracket \subseteq \mathbb{1}
\]
Equivalently, after picking your favourite singleton, for each object \(\mathfrak{X}\) we
have
\begin{equation}\label{eqn:interpretation_as_subset}
 \llbracket
    \exists e : \A,
    \forall x : X,
    \forall a \in \varphi(x),
    \Box ({e \odot a \downarrow} \land e \odot a \in \psi(x))
  \rrbracket_{\mathfrak{X}} \subseteq \{\star\}\tag{\(\dagger\)}
\end{equation}
with the condition that if a map \(f : \mathfrak{Y} \to \mathfrak{X}\) exists
then the interpretation at \(\mathfrak{X}\) is a subset of the interpretation at
\(\mathfrak{Y}\). If the subset~\ref{eqn:interpretation_as_subset} is inhabited,
i.e.\ it contains~\(\star\), then the statement ``holds'' at object
\(\mathfrak{X}\), otherwise it does not. Let us now start unfolding the
interpretation of this statement: we are after a condition for
\[
 \star \in \llbracket
    \exists e : \A,
    \forall x : X,
    \forall a \in \varphi(x),
    \Box ({e \odot a \downarrow} \land e \odot a \in \psi(x))
  \rrbracket_{\mathfrak{X}}
\]
Interpreting the existential, this holds if there exists some
\(\bar{e} : \A_{\mathfrak{X}}\) such that
\[
 \bar{e} \in \llbracket
    \forall x : X,
    \forall a \in \varphi(x),
    \Box ({e \odot a \downarrow} \land e \odot a \in \psi(x))
  \rrbracket_{\mathfrak{X}}
\]
For the next step, we take advantage of the fact that we have We will also take
advantage of the fact that we have
\(\forall_{\phi} \cdot \forall_{\psi} = \forall{\phi \cdot \psi}\) to interpret
both quantifiers at once, which will simplify things greatly. With that in mind,
the preceding holds if for all maps
\(f_{1} : \mathfrak{X}_{1} \to \mathfrak{X}\) and elements
\(\bar{x} : X_{\mathfrak{X}_{1}}\) and \(\bar{a} : \A_{\mathfrak{X}_{1}}\), if
\( \bar{a} \in \varphi_{\mathfrak{X}_{1}}(\bar{x}) \) then we must also have
\(
 (\bar{e} \cdot f_{1}\,,\,\bar{x}\,,\,\bar{a}) \in \llbracket
    \Box ({e \odot a \downarrow} \land e \odot a \in \psi(x))
  \rrbracket_{\mathfrak{X}_{1}}
\).
To simplify the consequent we see the modality \(\Box\) unfolds to the following:
it holds if there exists a sieve \(S\) on \(\mathcal{X}_{3}\) such that for all
\(f_{4} : \mathfrak{X}_{4} \to \mathfrak{X}_{3}\) in \(S\) we have
\begin{multline*}
 (\bar{e} \cdot f_{1}\cdot f_{2}\cdot f_{3} \cdot f_{4}, \bar{x} \cdot f_{2} \cdot f_{3} \cdot f_{4}, \bar{a} \cdot f_{3} \cdot f_{4}) \in \llbracket
    {e \odot a \downarrow} \land e \odot a \in \psi(x)
  \rrbracket_{\mathfrak{X}_{4}}
  \\
  =
  {\llbracket {e \odot a \downarrow} \rrbracket_{\mathfrak{X}_{4}}}
  \cap
  {\llbracket e \odot a \in \psi(x) \rrbracket_{\mathfrak{X}_{4}}}
\end{multline*}
Now,
\((\bar{e} \cdot f_{1}\cdot f_{2}\cdot f_{3} \cdot f_{4}, \bar{x} \cdot f_{2} \cdot f_{3} \cdot f_{4}, \bar{a} \cdot f_{3} \cdot f_{4})
\in
\llbracket {e \odot a \downarrow} \rrbracket_{\mathfrak{X}_{4}}\)
holds when the partial application
\(
(\bar{e} \cdot f_{1}\cdot f_{2}\cdot f_{3} \cdot f_{4})
\odot_{\mathfrak{X}_{4}}
(\bar{a} \cdot f_{3} \cdot f_{4})
\)
is defined in the usual set sense. As for
\(
(\bar{e} \cdot f_{1}\cdot f_{2}\cdot f_{3} \cdot f_{4}, \bar{x} \cdot f_{2} \cdot f_{3} \cdot f_{4}, \bar{a} \cdot f_{3} \cdot f_{4})
\in
{\llbracket e \odot a \in \psi(x) \rrbracket_{\mathfrak{X}_{4}}}
\),
it will hold when we have
\(
(\bar{e} \cdot f_{1}\cdot f_{2}\cdot f_{3} \cdot f_{4})
\odot_{\mathfrak{X}_{4}}
(\bar{a} \cdot f_{3} \cdot f_{4})
\in
\phi_{\mathfrak{X}_{4}}(\bar{x} \cdot f_{2} \cdot f_{3} \cdot f_{4})
\).

Truly we have now unfolded this definition past any level of readability or
usefulness. Putting a few assumptions


\section{A Realizability Tripos}

\todo[inline]{Define Heyting prealgebra}

\begin{defn}
  Given a presheaf \(\X\) we define the type of \definiendum{realizability
    predicates on \(\X\)} as the exponential \(\RealPred{\X} \).
  %
  We further define an ordering on realizability
  predicates~\(\varphi, \psi : \RealPred{\X}\) by
  %
  \[
    \varphi \leq \psi
    \IsDefined
    \IExists{e : \A}{
      \IForall{x : \X}{
        \IForall{a \in \varphi(x)}{
          \IBox{(
            \IAnd%
            {\ITotal{\IAp{e}{a}}}%
            {\IAp{e}{a} \in \psi(x)}
          )}
        }
      }
    }
  \]
  %
  We will say that \definiendum{\(e\) evidences \(\varphi \leq \psi\)} if
  it witnesses the existential quantifier above.
\end{defn}

%

\begin{lemm}
  For any presheaf \(\X\), \((\RealPred{\X}, \leq)\) is a preorder.
\end{lemm}
\begin{proof}
  Fix \(\varphi\), then \(\varphi \leq \varphi\) is always evidenced by
  \(\IIdentCode\).
  %
  Fix \(x : \X\) and \(a \in \varphi(x)\), since application of \(\IIdentCode\) is
  always well-defined and behaves as identity, it holds
  that~\(\ITotal{\IAp{\IIdentCode}{a}}\) and \({\IAp{\IIdentCode}{a}\in\varphi(x)}\).
  %
  This implies the necessary boxed statement to prove \(\leq\) reflexive.

  It is also transitive. Suppose that \(\varphi \leq \psi\) and
  \(\psi \leq \chi\) evidenced by \(e_{1}\) and \(e_{2}\) respectively.
  %
  Fixing \(x : \X\) and \(a \in \varphi(x)\), it follows by assumption that
  %
  \[
    \IBox{(
      \IAnd%
      {\ITotal{\IAp{e_{1}}{a}}}%
      {\IAp{e_{1}}{a} \in \psi(x)}
    )}
  \]
  %
  By monotonicity of \(\IBoxSym\) as well as our second assumption we may derive
  %
  \[
    \IBox{(
      \IAnd%
      {\ITotal{\IAp{e_{1}}{a}}}%
      {\IBox{[\IAnd%
        {\ITotal{\IAp{e_{2}}{(\IAp{e_{1}}{a})}}}%
        {\IAp{e_{2}}{(\IAp{e_{1}}{a})} \in \chi(x)}
      ]}}
    )}
  \]
  %
  As \(\IBoxSym\) is idempotent and preserves conjunctions we may simplify this
  to contain a single occurence of the modal box
  %
  \[
    \IBox{(
      \IAnd%
      {\ITotal{\IAp{e_{1}}{a}}}%
      {\IAnd%
        {\ITotal{\IAp{e_{2}}{(\IAp{e_{1}}{a})}}}%
        {\IAp{e_{2}}{(\IAp{e_{1}}{a})} \in \chi(x)}
      }
    )}
  \]
  %
  As needed, this is equivalent to saying that
  \(\IAp{\IAp{\ICompCode}{e_{1}}}{e_{2}}\) evidences \(\varphi \leq \chi\).
\end{proof}

\todo[inline]{prove it has finite meets}

\begin{lemm}
  For any presheaf \(\X\), the preorder \((\RealPred{\X}, \leq)\) has finite
  meets.
  %
  The top element is described by~\ref{defn:pred-top} and the meet of two
  realizability predicates~\(\varphi\) and \(\psi\) is described
  by~\ref{defn:pred-meet}:
  %
  \begin{gather}
    \TTop{\X}(x)
    \IsDefined
    \ISetComp{a:\A}{\ITop}
    \label{defn:pred-top}
    \\
    \TAnd{\varphi}{\psi}{\X}(x)
    \IsDefined
    \ISetComp{a:\A}{\IAnd%
      {\IAnd{\ITotal{\IAp{\IFstCode}{a}}}{\IAp{\IFstCode}{a}\in\varphi(x)}}%
      {\IAnd{\ITotal{\IAp{\ISndCode}{a}}}{\IAp{\ISndCode}{a}\in\psi(x)}}
    }
    \label{defn:pred-meet}
  \end{gather}
\end{lemm}
\begin{proof}
  We start by showing that \(\TTop{\X}\) is a greatest element.
  %
  Fixing \(\varphi:\RealPred{\X}\), then the identity code \(\IIdentCode\)
  evidences \(\varphi \leq \TTop{\X}\), as for all \(x : \X\) and
  \(a \in \varphi(x)\) we have \(\IAp{\IIdentCode}{a} = a\).

  Next we show \(\TAnd{\varphi}{\psi}{\X}\) is the meet of \(\varphi\) and
  \(\psi\) in \(\RealPred{\X}\).
  %
  The projection codes~\(\IFstCode\) and~\(\ISndCode\) evidence
  \(\TAnd{\varphi}{\psi}{\X}\leq\varphi\) and
  \(\TAnd{\varphi}{\psi}{\X}\leq\psi\) respectively:
  %
  if \(a \in \TAnd{\varphi}{\psi}{\X}\) then by definition we have
  \(\IAnd{\ITotal{\IAp{\IFstCode}{a}}}{\IAp{\IFstCode}{a}\in\varphi}\), as
  well as the corresponding for \(\ISndCode\) and \(\psi\).
  %
  As for the elimination rule, suppose we have \(\chi\leq\varphi\) and
  \(\chi\leq\psi\) evidenced by \(e_{1}\) and \(e_{2}\) respectively,
  then~%
  %
  \(
    e \IsDefined
    \IAbs{a}{\IPAp{\IPAp{\IPairCode}{(\IPAp{e_{1}}{a})}}{(\IPAp{e_{1}}{a})}}
  \)
  %
  evidences \(\chi\leq\TAnd{\varphi}{\psi}{\X}\).
  %
  We know this by specialising our assumptions on~\(e_{1}\) and~\(e_{2}\)
  to a fixed \(x:\X\) and \(a\in\chi(x)\) and commuting \(\IBoxSym\) with
  conjunctions, giving us
  %
  \[
    \IBox{
      (\IAnd{
        \IAnd{\ITotal{\IAp{e_{1}}{a}}}{\ITotal{\IAp{e_{2}}{a}}}
      }{
        \IAnd{\IAp{e_{1}}{a}\in\varphi(x)}{\IAp{e_{2}}{a}\in\psi(x)}
      })
    }
  \]
  %
  The first two conjuncts tell us that \(\ITotal{\IAp{e}{a}}\) and so by the
  specification of functional completeness we see that
  \(\IAp{e}{a} = \IAp{\IAp{\IPairCode}{(\IAp{e_{1}}{a})}}{(\IAp{e_{1}}{a})}\).
\end{proof}

\todo[inline]{prove it has finite joins}

\todo[inline]{prove it has Heyting implication}

% Hence we have proved these are heyting algebras

\todo[inline]{define pullback and show it is heyting algebra morphism}

\todo[inline]{prove left adjoint to pullback exists (as monotone map)}

\todo[inline]{prove right adjoint to pullback exists (as monotone map)}

\todo[inline]{prove beck-chevalley condition}

\todo[inline]{show there is a generic object}


\newpage
\printbibliography{}

\end{document}
