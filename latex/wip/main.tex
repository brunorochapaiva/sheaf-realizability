\documentclass[11pt]{article}

\newif\ifnotes
\notestrue % comment out to hide notes

\DeclareUnicodeCharacter{03A3}{\(\Sigma\)}
\DeclareUnicodeCharacter{03A0}{\(\Pi\)}
\DeclareFontFamily{U}{min}{}
\DeclareFontShape{U}{min}{m}{n}{<-> udmj30}{}
\newcommand\yo{\!\text{\usefont{U}{min}{m}{n}\symbol{'207}}\!}

\usepackage{geometry}
\geometry{
  a4paper,
  left=25mm,
  right=25mm,
  top=25mm,
}

% Packages for using unicode
\usepackage[utf8]{inputenc}
\usepackage[english]{babel}

\usepackage[T1]{fontenc}
%\usepackage[scaled=0.75]{DejaVuSansMono} % Good monospace font for code

\usepackage[
  backend=biber,
  mincrossrefs=999,
  style=numeric,
  doi=true,
  isbn=false,
  url=false,
  eprint=false,
]{biblatex}
\addbibresource{references.bib}

\usepackage[colorlinks=false]{hyperref}
\usepackage{aliascnt}
\usepackage{amsmath}
\usepackage{amssymb}
\usepackage{amsfonts}
\usepackage{amsthm}
\usepackage{cleveref}
\usepackage{newtxmath}
\usepackage{url}
\usepackage{tikz-cd}
\usepackage{quiver}
\usepackage{scalefnt}
\usepackage{microtype}
\usepackage{subfiles}
\usepackage{mathpartir}
\usepackage[bb=dsserif]{mathalpha}
\usepackage{bm}
\usepackage{mathtools}
\ifnotes%
\usepackage{todonotes}
\else
\usepackage[disable]{todonotes}
\fi
\usepackage{xspace}
\usepackage{multirow}
\usepackage{ulem}
\usepackage{contour}
\usepackage{xcolor}
\usepackage{booktabs}


\newtheorem{thrm}{Theorem}[section]
\newtheorem{lemm}[thrm]{Lemma}
\newtheorem{prop}[thrm]{Proposition}
\newtheorem{defn}[thrm]{Definition}
\newtheorem{remk}[thrm]{Remark}
\newtheorem{exam}[thrm]{Example}
\newtheorem{cons}[thrm]{Construction}
\newtheorem{coro}[thrm]{Corollary}
\newtheorem{nota}[thrm]{Notation}

%% Macros

%% Macro to superimpose two symbols
\makeatletter
\newcommand{\superimpose}[3][\mathord]{#1{\mathpalette\superimpose@{{#2}{#3}}}}
\newcommand{\superimpose@}[2]{\superimpose@@{#1}#2}
\newcommand{\superimpose@@}[3]{%
  \ooalign{%
    \hfil$\m@th#1#2$\hfil\cr
    \hfil$\m@th#1#3$\hfil\cr
  }%
}
\makeatother


%%%%%%%%%%%%%%%%%%%%%%%%%%%%%%%%%%%%%%%%%%%%%%%%%%%%%%%%%%%%%%%%%%%%%%%%%%%%%%%%
%% GENERAL MATHEMATICAL NOTATION %%%%%%%%%%%%%%%%%%%%%%%%%%%%%%%%%%%%%%%%%%%%%%%
%%%%%%%%%%%%%%%%%%%%%%%%%%%%%%%%%%%%%%%%%%%%%%%%%%%%%%%%%%%%%%%%%%%%%%%%%%%%%%%%

%% Used in the vernacular to mark a term being defined.
%% Currently, does nothing else besides making the term italic.
%% But it’s still nice to provide a layer of abstraction for this purpose.
%% It may be used, for example, to add things to the index.
\newcommand{\definiendum}[1]{\textbf{#1}}

\newcommand{\IsDefined}{\mathbin{\vcentcolon\equiv}}

%% General one letter notations
\newcommand{\A}{\ensuremath{\mathsf{A}}} % arbitrary PCA
\newcommand{\C}{\ensuremath{\mathcal{C}}} % arbitrary category
\newcommand{\E}{\ensuremath{\mathcal{E}}} % arbitrary topos
\renewcommand{\P}{\ensuremath{{P}}} % arbitrary poset (replaces paragraph symbol)
\renewcommand{\L}{\ensuremath{{L}}} % arbitrary lattice
\newcommand{\X}{\ensuremath{X}} % arbitrary sheaf
\newcommand{\N}{\mathbb{N}}

\newcommand{\Par}[1]{{#1}_\bot}
\newcommand{\ParTo}{\rightharpoonup}
\newcommand{\ParCat}[1]{{#1}^\ParTo}

\newcommand{\PresheafCat}{\hat{\C}}

%%%%%%%%%%%%%%%%%%%%%%%%%%%%%%%%%%%%%%%%%%%%%%%%%%%%%%%%%%%%%%%%%%%%%%%%%%%%%%%%
%% VERNACULAR COMMANDS %%%%%%%%%%%%%%%%%%%%%%%%%%%%%%%%%%%%%%%%%%%%%%%%%%%%%%%%%
%%%%%%%%%%%%%%%%%%%%%%%%%%%%%%%%%%%%%%%%%%%%%%%%%%%%%%%%%%%%%%%%%%%%%%%%%%%%%%%%

\newcommand{\SystemT}{\textsc{System T}}
\newcommand{\MLTT}{\textsc{MLTT}}
\newcommand{\Agda}{\textsc{Agda}}
\newcommand{\Boxtt}{\textsc{BoxTT}}

%%%%%%%%%%%%%%%%%%%%%%%%%%%%%%%%%%%%%%%%%%%%%%%%%%%%%%%%%%%%%%%%%%%%%%%%%%%%%%%%
%% INTERNAL %%%%%%%%%%%%%%%%%%%%%%%%%%%%%%%%%%%%%%%%%%%%%%%%%%%%%%%%%%%%%%%%%%%%
%%%%%%%%%%%%%%%%%%%%%%%%%%%%%%%%%%%%%%%%%%%%%%%%%%%%%%%%%%%%%%%%%%%%%%%%%%%%%%%%

%% Logical language
\newcommand{\ITop}{\top}

\newcommand{\INotSym}{\lnot}
\newcommand{\INot}[1]{\INotSym{#1}}

\newcommand{\IAndSym}{\land}
\newcommand{\IAnd}[2]{{#1}\IAndSym{#2}}

\newcommand{\IOrSym}{\lor}
\newcommand{\IOr}[2]{{#1}\IOrSym{#2}}

\newcommand{\IForallSym}{\forall}
\newcommand{\IForall}[2]{\IForallSym{#1}.{#2}}

\newcommand{\IExistsSym}{\exists}
\newcommand{\IExists}[2]{\IExistsSym{#1}.{#2}}

\newcommand{\IBoxSym}{\Box}
\newcommand{\IBox}[1]{\IBoxSym{#1}}

\newcommand{\ISetComp}[2]{\{{#1}\mid{#2}\}}

%% Types
\newcommand{\IPower}[1]{\mathcal{P}{#1}}
\newcommand{\IFun}[2]{{{#1}\to{#2}}}

% INTERNAL PCA

%% Partial application
\newcommand{\IApSym}{\cdot}
\newcommand{\IAp}[2]{{#1}\IApSym{#2}}

%% Polynomial application
\newcommand{\IPApSym}{\cdot}
\newcommand{\IPAp}[2]{{#1}\IPApSym{#2}}

%% Polynomial abstraction
\newcommand{\IAbs}[2]{\Lambda{#1}.\,{#2}}

%% Partial element is defined
\newcommand{\ITotal}[1]{{#1}\downarrow}

\newcommand{\IIdentCode}{\texttt{id}}

\newcommand{\ICompCode}{\texttt{comp}}

\newcommand{\IPairCode}{\texttt{pair}}
\newcommand{\IFstCode}{\texttt{p}_0}
\newcommand{\ISndCode}{\texttt{p}_1}


%% Realizability Predicates
\newcommand{\RealPred}[1]{\IFun{#1}{\IPower{\A}}}

%% Tripos language

\newcommand{\TTop}[1]{\top}

\newcommand{\TAndSym}[1]{\land}
\newcommand{\TAnd}[3]{{#1}\TAndSym{#3}{#2}}


%--------------------------------------------------------------------------------
% Document proper
\begin{document}

\title{Realizability with Sheaves}
\author{Bruno da Rocha Paiva}
\maketitle

We begin a treatment of realizability internal to categories of sheaves.

\newpage
\section{Internal Language of a Presheaf Topos}

A typical course in mathematical logic may start by looking at formal systems
such as natural deduction or the sequent calculus, giving rise to a notion of
truth, namely the provability of formulas. A natural next step is to then look
at the semantics of said proof system. These semantics can vary significantly
depending on the chosen logic such as boolean valuations for propositional
logic, Kripke frames for modal logic, \(\mathcal{L}\)-structures for predicate
logic, etc. Regardless, in each case we can consider whether a formula is valid,
i.e.\ true in every model and this gives rise to yet to a second notion of
truth. Remarkably, these two notions of truth often coincide, provided the right
proof system and semantics are chosen. While provability quantifies
existentially over proofs, asking whether a formal proof exists, validity is a
universal statement asking that the interpretation of a formula hold in all
models. This duality is powerful: to show a formula \(\phi\) is not provable by
proof-theoretic means is difficult. Typical methods start by identifying some
property of proofs \(P\) such that any proof of any formula \(\psi\) can be
transformed to another proof of \(\psi\) satisfying \(P\). From here, if we can
show that no proof of \(\phi\) satisfying \(P\) exists, then that must mean no
proof exists in general. Typical examples of properties \(P\) can be that the
proof does not make use of the \textbf{Cut} rule in sequent calculus or
alternatively for natural deduction that there exist no \(\beta\)-redexs. While
these are interesting and useful results to have, in this case the semantics
offers an easier route. To show \(\phi\) is not valid it suffices to demonstrate
a model in which the interpretation of \(\phi\) does not hold and this is
generally easier than the syntactic route.

At a first glance the power of this duality seems to be in dispatching of syntax
and work purely with models. As we move to more and more complicated languages,
going beyond first-order logic to its higher-order counterparts, then the needle
begins to move back. In particular, consider (your favourite version of)
Martin-L\"of Type Theory (\MLTT{}) with \(\Pi\)-types and \(\Sigma\)-types.
Semantics for \MLTT{} are usually categorically, either through categories with
families, contextual categories, natural models, the list goes on. Fixing your
favourite notion of model, categories of presheaves \(\PresheafCat\) generally
give rise to a model of dependent type theory. That means if we prove some
statement in \MLTT{}, then by taking the interpretation of said statement in
\(\PresheafCat\) we will get a true statement about presheaves over \(\C\).
Working with presheaves can be quite labourious and often becomes an exercise
in bookkeeping of indices.

\subsection{Exponentials}

Given two types \(A, B\), the type of functions from \(A\) to \(B\),
denoted \(\IFun{A}{B}\), behaves like the usual function type in type theory.
%
As for its categorical semantics we have the following

\begin{defn}
  Given two objects \(A, B\), their exponential is an object \(\IFun{A}{B}\)
  along with a morphism \(\EEval : A \times \IFun{A}{B} \to B\) such that
  for all \(f : C \times A \to B\) there exists a unique
  \(\tilde{f}: C \to \IFun{A}{B}\) making the following triangle commutes
  %
  \[\begin{tikzcd}
    {C \times \IFun{A}{B}} && B \\
    \\
    {C\times A}
    \arrow["{\EEval}", from=1-1, to=1-3]
    \arrow["{C\times\tilde{f}}", from=3-1, to=1-1]
    \arrow["f"', from=3-1, to=1-3]
  \end{tikzcd}\]
\end{defn}

\subsubsection{Exponentials of Presheaves}

The exponential of two presheaves \(X,Y\) is a presheaf
\(\IFun{X}{Y}\) with action on objects given by
%
\[
  (\IFun{X}{Y})_{w} \IsDefined \mathsf{Nat}(\yo(w) \times X, Y)
\]
%
and action on morphisms defined as follows
%
\[\begin{array}{l}
  \IFun{X}{Y}_{f:u \to w} :
    \mathsf{Nat}(\yo(w) \times X, Y) \to \mathsf{Nat}(\yo(u) \times X, Y)\\
  (\IFun{X}{Y}_{f:u \to w}(\alpha))_{v}(g : v \to u, x : X_{v})
    \IsDefined \alpha_{v}(fg, x)
\end{array}\]

The evaluation map is a natural transformation
\(\EEval : X \times \IFun{X}{Y} \to Y\) defined for each object
\(w\) by
%
\[\begin{array}{l}
  \EEval_{w} : X \times \mathsf{Nat}(\yo(w) \times X, Y) \to Y\\
  \EEval_{w}(x, \alpha) \IsDefined \alpha_{w}(\mathsf{id}, x)
\end{array}\]
%
Naturality of \(\EEval\) means for all \(x : X_{w}\),
\(\alpha : \mathsf{Nat}(\yo(w) \times X, Y)\) and \(f : v \to w\) we have
\[
  Y_{f}(\alpha_{w}(\mathsf{id}, x)) =
  Y_{f}(\EEval_{w}(x, \alpha)) =
  \EEval_{v}(X_{f}(x), \IFun{X}{Y}_{f}(\alpha)) =
  (\IFun{X}{Y}_{f}(\alpha))_{v}(\mathsf{id}, X_{f}(x)) =
  \alpha_{v}(f, X_{f}(x))
\]

%\newpage
%\section{Internal PCAs}
%
%In set-based mathematics, partial functions \(f : X \ParTo Y\) are usually
%defined as a subset \(\text{dom}(f) \subseteq X\) and a total function
%\(f : \text{dom}(f) \to Y\) on said subset. With such a definition, the
%composition of two partial functions \(f : X \ParTo Y\) and \(g : Y \ParTo Z\)
%will have domain \(\text{dom}(g \circ f) = \text{im}(f) \cap \text{dom}(g)\), on
%which the composite \(g \circ f\) is total. Associativity of function
%composition comes down to associativity of subset intersections and as a total
%function is ``trivially'' partial we also have identity maps satisfying the unit
%laws. With this, we have just defined a category of sets and partial functions
%between them. A similar construction can be done for arbitrary categories under
%the mild condition that pullbacks along monomorphisms exist. Said pullbacks will
%provide composition of partial morphisms, leading us to the final obstruction,
%which is that pullbacks are only associative up to (unique) isomorphism. This is
%remedied by identifying partial maps up to isomorphism of the domains as we will
%make explicit later.
%
%\begin{defn}
%  Given a category \(\C\) with pullbacks we define the \definiendum{category of
%    partial maps} in \(\C\), denoted \(\ParCat{\C}\), according to the following:
%  \begin{itemize}
%    \item objects of \(\ParCat{\C}\) are exactly those of \(\C\).
%    \item given objects \(X,Y\), their hom-set \(\ParCat{\C}(X,Y)\) is given by
%      equivalence classes of spans \(X \xhookleftarrow{m} S \xrightarrow{f} Y\)
%      with \(m\) monic. Two such spans
%      \(X \xhookleftarrow{m} S \xrightarrow{f} Y\) and
%      \(X \xhookleftarrow{n} P \xrightarrow{g} Y\) are equivalent if there
%      exists an isomorphism \(f : S \cong P\) such that the following commutes
%      \[\begin{tikzcd}
%          && S \\
%          X &&&& Y \\
%          && P \arrow["m"', hook, from=1-3, to=2-1] \arrow["f", from=1-3, to=2-5] \arrow["f"', from=1-3, to=3-3] \arrow["n", hook', from=3-3, to=2-1] \arrow["g"', from=3-3, to=2-5]
%        \end{tikzcd}\]
%    \item Composition of two morphisms
%      \(X \xhookleftarrow{m} S \xrightarrow{f} Y\) and
%      \(Y \xhookleftarrow{n} P \xrightarrow{g} Z\) is given by the pullback
%      \[\begin{tikzcd}
%          X \\
%          S & Y \\
%          {S\times_YP} & P & Z \arrow["m", hook', from=2-1, to=1-1] \arrow["f"', from=2-1, to=2-2] \arrow[from=3-1, to=2-1] \arrow["\lrcorner"{anchor=center, pos=0.125, rotate=90}, draw=none, from=3-1, to=2-2] \arrow[from=3-1, to=3-2] \arrow["n", hook', from=3-2, to=2-2] \arrow["g"', from=3-2, to=3-3]
%        \end{tikzcd}\]
%    \item Identity maps are given by the span consisting of two identity maps.
%  \end{itemize}
%\end{defn}
%
%The associative and unit laws can be checked to follow from the uniqueness (up
%to isomorphism) of pullbacks. Working through this definition for the category
%of sets we recover a category isomorphic to our original description. In the
%specific case of sets we were able to avoid worrying about equivalence classes
%by using subsets instead of subobjects and subset intersections instead of
%pullbacks. Similarly, we can give a more concrete description for the category
%of partial maps of sheaves.

\section{A Realizability Tripos}

\todo[inline]{Define Heyting prealgebra}

\begin{defn}
  Given a presheaf \(\X\) we define the type of \definiendum{realizability
    predicates on \(\X\)} as the exponential \(\RealPred{\X} \).
  %
  We further define an ordering on realizability
  predicates~\(\varphi, \psi : \RealPred{\X}\) by
  %
  \[
    \varphi \leq \psi
    \IsDefined
    \IExists{e : \A}{
      \IForall{x : \X}{
        \IForall{a \in \varphi(x)}{
          \IBox{(
            \IAnd%
            {\ITotal{\IAp{e}{a}}}%
            {\IAp{e}{a} \in \psi(x)}
          )}
        }
      }
    }
  \]
  %
  We will say that \definiendum{\(e\) evidences \(\varphi \leq \psi\)} if
  it witnesses the existential quantifier above.
\end{defn}

%

\begin{lemm}
  For any presheaf \(\X\), \((\RealPred{\X}, \leq)\) is a preorder.
\end{lemm}
\begin{proof}
  Fix \(\varphi\), then \(\varphi \leq \varphi\) is always evidenced by
  \(\IIdentCode\).
  %
  Fix \(x : \X\) and \(a \in \varphi(x)\), since application of \(\IIdentCode\) is
  always well-defined and behaves as identity, it holds
  that~\(\ITotal{\IAp{\IIdentCode}{a}}\) and \({\IAp{\IIdentCode}{a}\in\varphi(x)}\).
  %
  This implies the necessary boxed statement to prove \(\leq\) reflexive.

  It is also transitive. Suppose that \(\varphi \leq \psi\) and
  \(\psi \leq \chi\) evidenced by \(e_{1}\) and \(e_{2}\) respectively.
  %
  Fixing \(x : \X\) and \(a \in \varphi(x)\), it follows by assumption that
  %
  \[
    \IBox{(
      \IAnd%
      {\ITotal{\IAp{e_{1}}{a}}}%
      {\IAp{e_{1}}{a} \in \psi(x)}
    )}
  \]
  %
  By monotonicity of \(\IBoxSym\) as well as our second assumption we may derive
  %
  \[
    \IBox{(
      \IAnd%
      {\ITotal{\IAp{e_{1}}{a}}}%
      {\IBox{[\IAnd%
        {\ITotal{\IAp{e_{2}}{(\IAp{e_{1}}{a})}}}%
        {\IAp{e_{2}}{(\IAp{e_{1}}{a})} \in \chi(x)}
      ]}}
    )}
  \]
  %
  As \(\IBoxSym\) is idempotent and preserves conjunctions we may simplify this
  to contain a single occurence of the modal box
  %
  \[
    \IBox{(
      \IAnd%
      {\ITotal{\IAp{e_{1}}{a}}}%
      {\IAnd%
        {\ITotal{\IAp{e_{2}}{(\IAp{e_{1}}{a})}}}%
        {\IAp{e_{2}}{(\IAp{e_{1}}{a})} \in \chi(x)}
      }
    )}
  \]
  %
  As needed, this is equivalent to saying that
  \(\IAp{\IAp{\ICompCode}{e_{1}}}{e_{2}}\) evidences \(\varphi \leq \chi\).
\end{proof}

%\todo[inline]{prove it has finite meets}

\begin{lemm}
  For any presheaf \(\X\), the preorder \((\RealPred{\X}, \leq)\) has finite
  meets.
  %
  The top element is described by~\ref{defn:pred-top} and the meet of two
  realizability predicates~\(\varphi\) and \(\psi\) is described
  by~\ref{defn:pred-meet}:
  %
  \begin{gather}
    \TTop{\X}(x)
    \IsDefined
    \ISetComp{a:\A}{\ITop}
    \label{defn:pred-top}
    \\
    \TAnd{\varphi}{\psi}{\X}(x)
    \IsDefined
    \ISetComp{a:\A}{\IAnd%
      {\IAnd{\ITotal{\IAp{\IFstCode}{a}}}{\IAp{\IFstCode}{a}\in\varphi(x)}}%
      {\IAnd{\ITotal{\IAp{\ISndCode}{a}}}{\IAp{\ISndCode}{a}\in\psi(x)}}
    }
    \label{defn:pred-meet}
  \end{gather}
\end{lemm}
\begin{proof}
  We start by showing that \(\TTop{\X}\) is a greatest element.
  %
  Fixing \(\varphi:\RealPred{\X}\), then the identity code \(\IIdentCode\)
  evidences \(\varphi \leq \TTop{\X}\), as for all \(x : \X\) and
  \(a \in \varphi(x)\) we have \(\IAp{\IIdentCode}{a} = a\).

  Next we show \(\TAnd{\varphi}{\psi}{\X}\) is the meet of \(\varphi\) and
  \(\psi\) in \(\RealPred{\X}\).
  %
  The projection codes~\(\IFstCode\) and~\(\ISndCode\) evidence
  \(\TAnd{\varphi}{\psi}{\X}\leq\varphi\) and
  \(\TAnd{\varphi}{\psi}{\X}\leq\psi\) respectively:
  %
  if \(a \in \TAnd{\varphi}{\psi}{\X}\) then by definition we have
  \(\IAnd{\ITotal{\IAp{\IFstCode}{a}}}{\IAp{\IFstCode}{a}\in\varphi}\), as
  well as the corresponding for \(\ISndCode\) and \(\psi\).
  %
  As for the elimination rule, suppose we have \(\chi\leq\varphi\) and
  \(\chi\leq\psi\) evidenced by \(e_{1}\) and \(e_{2}\) respectively,
  then~%
  %
  \(
    e \IsDefined
    \IAbs{a}{\IPAp{\IPAp{\IPairCode}{(\IPAp{e_{1}}{a})}}{(\IPAp{e_{1}}{a})}}
  \)
  %
  evidences \(\chi\leq\TAnd{\varphi}{\psi}{\X}\).
  %
  We know this by specialising our assumptions on~\(e_{1}\) and~\(e_{2}\)
  to a fixed \(x:\X\) and \(a\in\chi(x)\) and commuting \(\IBoxSym\) with
  conjunctions, giving us
  %
  \[
    \IBox{
      (\IAnd{
        \IAnd{\ITotal{\IAp{e_{1}}{a}}}{\ITotal{\IAp{e_{2}}{a}}}
      }{
        \IAnd{\IAp{e_{1}}{a}\in\varphi(x)}{\IAp{e_{2}}{a}\in\psi(x)}
      })
    }
  \]
  %
  The first two conjuncts tell us that \(\ITotal{\IAp{e}{a}}\) and so by the
  specification of functional completeness we see that
  \(\IAp{e}{a} = \IAp{\IAp{\IPairCode}{(\IAp{e_{1}}{a})}}{(\IAp{e_{1}}{a})}\).
\end{proof}

\todo[inline]{prove it has finite joins}

\todo[inline]{prove it has Heyting implication}

% Hence we have proved these are heyting algebras

\todo[inline]{define pullback and show it is heyting algebra morphism}

\todo[inline]{prove left adjoint to pullback exists (as monotone map)}

\todo[inline]{prove right adjoint to pullback exists (as monotone map)}

\todo[inline]{prove beck-chevalley condition}

\todo[inline]{show there is a generic object}

\subsection{Presheaves with a Lawvere-Tierney Topology}

Fix two realizability predicates \(\phi,\psi : \RealPred{\X}\), that is global
elements \(\phi,\psi : 1 \to \RealPred{\X}\).
%
We will unfold the denotation of \(\phi \leq \psi\) in the case of a preasheaf
category with some Lawvere-Tierney topology \(j : \Omega \to \Omega\).
%
We are, of course, interested in the following mono
%
\[
  \llbracket
    \IExists{e : \A}{
      \IForall{x : \X}{
        \IForall{a \in \varphi(x)}{
          \IBox{(
            \IAnd%
            {\ITotal{\IAp{e}{a}}}%
            {\IAp{e}{a} \in \psi(x)}
          )}
        }
      }
    }
  \rrbracket : \cdot \hookrightarrow 1
\]
%
We interpret the existential along the projection \(\pi_{\A} : 1 \times \A \to 1\)
giving us
%
\[
  \exists_{\pi_{\A}}(\underbrace{\llbracket
    \IForall{x : \X}{
      \IForall{a \in \varphi(x)}{
        \IBox{(
          \IAnd%
          {\ITotal{\IAp{e}{a}}}%
          {\IAp{e}{a} \in \psi(x)}
        )}
      }
    }
  \rrbracket}_{\text{subobject of }1 \times A})
\]
%
We interpret the universal along the projection
\(\pi_{\X} : (1 \times \A) \times \X \to 1 \times \A\) giving
%
\[
  \exists_{\pi_{\A}}\circ\forall_{\pi_{\X}}(\underbrace{\llbracket
    \IForall{a \in \varphi(x)}{
      \IBox{(
        \IAnd%
        {\ITotal{\IAp{e}{a}}}%
        {\IAp{e}{a} \in \psi(x)}
      )}
    }
  \rrbracket}_{\text{subobject of }(1 \times A) \times \X})
\]
%
Next we need to interpret the relative universal quantifier.
%
For this, the denotation of the open term
\(e : \A, x : \X \vdash \phi(x) : \IPower{\A}\) gives us a mono
%
\(
  m :
  S \hookrightarrow ((1 \times \A) \times \X) \times \A
\)
%
and composing this with the projection
\(\pi'_{\A} : ((1 \times \A) \times \X) \times \A \to (1 \times \A) \times \X\)
will give us the map along which we will interpret the quantifier.
%
\[
  \exists_{\pi_{\A}}\circ
  \forall_{\pi_{\X}}\circ
  \forall_{\pi'_{\A}\circ m}(\underbrace{\llbracket
    \IBox{(
      \IAnd%
      {\ITotal{\IAp{e}{a}}}%
      {\IAp{e}{a} \in \psi(x)}
    )}
  \rrbracket}_{\text{subobject of } S})
\]
%
Quantifiers are functorial so this is equivalent to the following
statement with a single universal
%
\[
  \exists_{\pi_{\A}}\circ
  \forall_{\pi_{\X}\circ\pi'_{\A}\circ m}(\underbrace{\llbracket
    \IBox{(
      \IAnd%
      {\ITotal{\IAp{e}{a}}}%
      {\IAp{e}{a} \in \psi(x)}
    )}
  \rrbracket}_{\text{subobject of } S})
\]
%
The modality is interpeted by taking the closure of the subobjects arising
out of the Lawvere-Tierney topology \(j:\Omega\to\Omega\).
%
\[
  \exists_{\pi_{\A}}\circ
  \forall_{\pi_{\X}}\circ
  \forall_{\pi'_{\A}\circ m}
  (\overline{\underbrace{\llbracket
      \IAnd%
      {\ITotal{\IAp{e}{a}}}%
      {\IAp{e}{a} \in \psi(x)}
  \rrbracket}_{\text{subobject of } S}})
\]
%
Conjunction is interpreted by the meet of subobjects. Such meets are computed
by pointwise intersections, but we can safely ignore that for now.
%
\[
  \exists_{\pi_{\A}}\circ
  \forall_{\pi_{\X}\circ\pi'_{\A}\circ m}
  \underset{\text{both subobjects of } S}%
  {(\overline{\underbrace{\llbracket \ITotal{\IAp{e}{a}}\rrbracket}
   \wedge
   \underbrace{\llbracket\IAp{e}{a} \in \psi(x)\rrbracket}
  })}
\]
%
At this point let us see how to interpret the particular terms mentioned.
%
We interpret the term \(e : \A, x : \X \vdash \phi(x) : \IPower{\A} \) as a
morphism \((1 \times \A) \times \X \to \IPower{\A}\) given by preocomposing
\(\EEval : X \times \RealPred{\X}\) with the pairing of the projection
\((1 \times \A) \times \X \to \X\) and of the map
\((1 \times \A) \times \X \to \RealPred{\X}\) which we get by weakening
\(\phi : 1 \to \RealPred{\X}\).
%
For clarity, the natural transformation \(\EEval\) is given by
%
\begin{align*}
  &\EEval_{w} : \X_{w} \times \mathsf{Nat}(\yo(w) \times \X, \IPower{\A})
                \to \IPower{\A}_{w}\\
  &\EEval_{w}(x, \alpha) = \alpha_{w}(\mathsf{id}, x)
\end{align*}
%
meaning that the denotation \(\llbracket\phi(x)\rrbracket\) is given by
%
\begin{align*}
  &\llbracket\phi(x)\rrbracket_{w} : (1 \times \A) \times \X \to \IPower{\A}\\
  &\llbracket\phi(x)\rrbracket_{w}((\star, a), x) =
    \EEval_{w}(x, \phi_{w}) = \phi_{w}(\mathsf{id}, x)
\end{align*}
%
Naturality in this case means that for any \(f : u \to w\), \(g : v \to u\)
and \(x : \X_{u}\) we have
\[
  (\alpha_{u}(f, x))\{g\} = \alpha_{v}(fg, x\{g\})
\]


\[
  ([\alpha_{v}(fg, x\{g\})]_{u}(f, a)) = \alpha_{v}(fg, x\{g\})
\]

We can now define the mono \(m\) from \(\llbracket\phi(x)\rrbracket\) as
the mono corresponding to the characteristic function
\(\chi_{m} : (1 \times \A) \times \X \to \Omega\) arising by precomposing
with the interpretations of \(e : \A, x : \X \vdash a : \A  \)
and the weakening of \(e : \A, x : \X \vdash \phi(x) : \IPower{\A}\).
%
In the end this means that we have
\[
  (((\star, e), x), a) \in \mathsf{domain}(m)_{w}
  \iff
  w \in [\phi_{w}(\mathsf{id}, x)]_{w}(\mathsf{id}, a)
\]

\dots

And this leads us to the following external definition of the ordering
on realizability predicates.

\begin{defn}
  We say that \(\phi \leq \psi\) at world \(w\) if there exists a code
  \(e : \A_{w}\) such that for all maps \(f : u \leq w\), elements
  \(x : \X_{u}\) and codes \(a : \A_{u}\), if
  %
  \[
    \mathsf{id}_{u} \in [\phi_{u}(\mathsf{id}_{u}, x\{f\})]_{u}(\mathsf{id}_{u},a)
  \]
  %
  then there exists a cover \(\mathcal{V}\) of \(u\) such that for all
  \((g : v \to u) \in \mathcal{V}\) we have
  %
  \[\begin{array}{ccc}
  e\{fg\} \cdot_{v} a\{g\} \downarrow &\text{ and }&
  \mathsf{id}_{v} \in [\phi_{v}(\mathsf{id}_{v}, x\{fg\})]_{v}(\mathsf{id}_{v},e\{fg\} \cdot_{v} a\{g\}).
  \end{array}\]
\end{defn}

In the case that the underlying category is a poset
%
\begin{defn}[Over a poset]
  We say that \(\phi \leq \psi\) at world \(w\) if there exists a code
  \(e : \A_{w}\) such that for all extensions \(u \leq w\), elements
  \(x : \X_{u}\) and codes \(a : \A_{u}\), if
  %
  \[
    u \in [\phi_{u}(u\leq u, x|_{u})]_{u}(u \leq u,a)
  \]
  %
  then there exists a cover \(\mathcal{V}\) of \(u\) such that for all \(v \in \mathcal{V}\) we
  have
  %
  \[\begin{array}{ccc}
    e|_{v} \cdot_{v} a|_{v} \downarrow &\text{ and }&
    v \in [\phi_{v}(v\leq v, x|_{v})]_{v}(v \leq v,e|_{v} \cdot_{v} a|_{v})
  \end{array}\]
\end{defn}

And further assuming that \(\A\) is a constant presheaf we get
%
\begin{defn}[Over a poset with constant \(\A\)]
  We say that \(\phi \leq \psi\) at world \(w\) if there exists a code
  \(e : \A\) such that for all extensions \(u \leq w\), elements
  \(x : \X_{u}\) and codes \(a : \A\), if
  %
  \[
    u \in [\phi_{u}(u\leq u, x|_{u})]_{u}(u \leq u,a)
  \]
  %
  then there exists a cover \(\mathcal{V}\) of \(u\) such that for all \(v \in \mathcal{V}\) we
  have
  %
  \[\begin{array}{ccc}
    e \cdot_{v} a \downarrow &\text{ and }&
    v \in [\phi_{v}(v\leq v, x|_{v})]_{v}(v \leq v,e \cdot_{v} a)
  \end{array}\]
\end{defn}

We can also work with uncurried predicates, so now we have
\(\phi,\psi : \IPower{(\X\times\A)}\), the previous assumptions and if we
also omit the proofs that specific worlds are extensions of others, then
we get the following definition, which is what we have been working with
already:
%
\begin{defn}[Lots of simplifications]
  We say that \(\phi \leq \psi\) at world \(w\) if there exists a code
  \(e : \A\) such that for all extensions \(u \leq w\), elements
  \(x : \X_{u}\) and codes \(a : \A\), if
  %
  \[
    u \in \phi_{u}(x|_{u}, a)
  \]
  %
  then there exists a cover \(\mathcal{V}\) of \(u\) such that for all \(v \in \mathcal{V}\) we
  have
  %
  \[\begin{array}{ccc}
    e \cdot_{v} a \downarrow &\text{ and }&
    v \in \phi_{v}(x|_{v}, e \cdot_{v} a)
  \end{array}\]
\end{defn}

\newpage
\printbibliography{}

\end{document}
