% easychair.tex,v 3.5 2017/03/15

\documentclass{easychair}
%\documentclass[EPiC]{easychair}
%\documentclass[EPiCempty]{easychair}
%\documentclass[debug]{easychair}
%\documentclass[verbose]{easychair}
%\documentclass[notimes]{easychair}
%\documentclass[withtimes]{easychair}
%\documentclass[a4paper]{easychair}
%\documentclass[letterpaper]{easychair}

\usepackage{doc}

% use this if you have a long article and want to create an index
% \usepackage{makeidx}

% In order to save space or manage large tables or figures in a
% landcape-like text, you can use the rotating and pdflscape
% packages. Uncomment the desired from the below.
%
% \usepackage{rotating}
% \usepackage{pdflscape}

\usepackage{amsmath}
\usepackage{amssymb}
\usepackage{amsfonts}
\usepackage{amsthm}
\usepackage{newtxmath}
\usepackage{url}
\usepackage{tikz-cd}
\usepackage{quiver}
\usepackage{scalefnt}
\usepackage{microtype}
\usepackage{subfiles}
\usepackage{mathpartir}
\usepackage[bb=dsserif]{mathalpha}
\usepackage{bm}
\usepackage{mathtools}
\usepackage{cleveref}
\usepackage{xspace}
\usepackage{multirow}
\usepackage{ulem}
\usepackage{contour}

\newtheorem{thrm}{Theorem}
\newtheorem{lemm}[thrm]{Lemma}
\newtheorem{prop}[thrm]{Proposition}
\newtheorem{defn}[thrm]{Definition}
\newtheorem{remk}[thrm]{Remark}
\newtheorem{exam}[thrm]{Example}
\newtheorem{cons}[thrm]{Construction}
\newtheorem{coro}[thrm]{Corollary}
\newtheorem{nota}[thrm]{Notation}

%% Macros

%% Macro to superimpose two symbols
\makeatletter
\newcommand{\superimpose}[3][\mathord]{#1{\mathpalette\superimpose@{{#2}{#3}}}}
\newcommand{\superimpose@}[2]{\superimpose@@{#1}#2}
\newcommand{\superimpose@@}[3]{%
  \ooalign{%
    \hfil$\m@th#1#2$\hfil\cr
    \hfil$\m@th#1#3$\hfil\cr
  }%
}
\makeatother

%%%%%%%%%%%%%%%%%%%%%%%%%%%%%%%%%%%%%%%%%%%%%%%%%%%%%%%%%%%%%%%%%%%%%%%%%%%%%%%%
%% GENERAL MATHEMATICAL NOTATION %%%%%%%%%%%%%%%%%%%%%%%%%%%%%%%%%%%%%%%%%%%%%%%
%%%%%%%%%%%%%%%%%%%%%%%%%%%%%%%%%%%%%%%%%%%%%%%%%%%%%%%%%%%%%%%%%%%%%%%%%%%%%%%%

%% Used in the vernacular to mark a term being defined.
%% Currently, does nothing else besides making the term italic.
%% But it’s still nice to provide a layer of abstraction for this purpose.
%% It may be used, for example, to add things to the index.
\newcommand{\definiendum}[1]{\textbf{#1}}

\newcommand{\IsDefined}{\mathbin{\vcentcolon\equiv}}

%% General one letter notations
\newcommand{\A}{\ensuremath{\mathsf{A}}} % arbitrary PCA
\newcommand{\C}{\ensuremath{\mathcal{C}}} % arbitrary category
\renewcommand{\P}{\ensuremath{{P}}} % arbitrary poset (replaces paragraph symbol)
\renewcommand{\L}{\ensuremath{{L}}} % arbitrary lattice
\newcommand{\X}{\ensuremath{X}} % arbitrary sheaf
\newcommand{\E}{\mathcal{E}} % arbitrary topos
\newcommand{\Ej}{\mathcal{E}_\IBoxSym} % arbitrary topos of sheaves

\newcommand{\Par}[1]{{#1}_\bot}
\newcommand{\ParTo}{\rightharpoonup}
\newcommand{\ParCat}[1]{{#1}^\ParTo}

\newcommand{\PresheafCat}{\hat{\C}}

\newcommand{\Sub}[1]{\mathsf{Sub}({#1})}

%%%%%%%%%%%%%%%%%%%%%%%%%%%%%%%%%%%%%%%%%%%%%%%%%%%%%%%%%%%%%%%%%%%%%%%%%%%%%%%%
%% VERNACULAR COMMANDS %%%%%%%%%%%%%%%%%%%%%%%%%%%%%%%%%%%%%%%%%%%%%%%%%%%%%%%%%
%%%%%%%%%%%%%%%%%%%%%%%%%%%%%%%%%%%%%%%%%%%%%%%%%%%%%%%%%%%%%%%%%%%%%%%%%%%%%%%%

\newcommand{\SystemT}{\textsc{System T}}
\newcommand{\MLTT}{\textsc{MLTT}}
\newcommand{\Agda}{\textsc{Agda}}
\newcommand{\Boxtt}{\textsc{BoxTT}}

%%%%%%%%%%%%%%%%%%%%%%%%%%%%%%%%%%%%%%%%%%%%%%%%%%%%%%%%%%%%%%%%%%%%%%%%%%%%%%%%
%% DECORATION %%%%%%%%%%%%%%%%%%%%%%%%%%%%%%%%%%%%%%%%%%%%%%%%%%%%%%%%%%%%%%%%%%
%%%%%%%%%%%%%%%%%%%%%%%%%%%%%%%%%%%%%%%%%%%%%%%%%%%%%%%%%%%%%%%%%%%%%%%%%%%%%%%%

\setlength{\ULdepth}{3pt}
\contourlength{1pt}
\newcommand{\underlinehelper}[3]{
  \colorbox{#2}{
  #1{\phantom{#3}}%
  \llap{\contour{#2}{#3}}%
  }
}
\newcommand{\hla}[1]{\underlinehelper{\uline}{pink}{#1}}
\newcommand{\hlb}[1]{\underlinehelper{\uwave}{yellow}{#1}}
\newcommand{\hlc}[1]{\underlinehelper{\uuline}{pink}{#1}}

%%%%%%%%%%%%%%%%%%%%%%%%%%%%%%%%%%%%%%%%%%%%%%%%%%%%%%%%%%%%%%%%%%%%%%%%%%%%%%%%
%% EXTERNAL %%%%%%%%%%%%%%%%%%%%%%%%%%%%%%%%%%%%%%%%%%%%%%%%%%%%%%%%%%%%%%%%%%%%
%%%%%%%%%%%%%%%%%%%%%%%%%%%%%%%%%%%%%%%%%%%%%%%%%%%%%%%%%%%%%%%%%%%%%%%%%%%%%%%%

%% Eval map
\newcommand{\EEval}{\mathsf{ev}}

%%%%%%%%%%%%%%%%%%%%%%%%%%%%%%%%%%%%%%%%%%%%%%%%%%%%%%%%%%%%%%%%%%%%%%%%%%%%%%%%
%% INTERNAL %%%%%%%%%%%%%%%%%%%%%%%%%%%%%%%%%%%%%%%%%%%%%%%%%%%%%%%%%%%%%%%%%%%%
%%%%%%%%%%%%%%%%%%%%%%%%%%%%%%%%%%%%%%%%%%%%%%%%%%%%%%%%%%%%%%%%%%%%%%%%%%%%%%%%

% Logical language

\newcommand{\ITop}{\top}

\newcommand{\IBot}{\bot}

\newcommand{\INotSym}{\lnot}
\newcommand{\INot}[1]{\INotSym{#1}}

\newcommand{\IAndSym}{\land}
\newcommand{\IAnd}[2]{{#1}\IAndSym{#2}}

\newcommand{\IOrSym}{\lor}
\newcommand{\IOr}[2]{{#1}\IOrSym{#2}}

\newcommand{\IImpliesSym}{\Rightarrow}
\newcommand{\IImplies}[2]{{#1}\IImpliesSym{#2}}

\newcommand{\IForallSym}{\forall}
\newcommand{\IForall}[2]{\IForallSym{#1}.\,{#2}}

\newcommand{\IExistsSym}{\exists}
\newcommand{\IExists}[2]{\IExistsSym{#1}.\,{#2}}

\newcommand{\IBoxSym}{\Box}
\newcommand{\IBox}[1]{\IBoxSym{#1}}

\newcommand{\ISetComp}[2]{\{{#1}\mid{#2}\}}

\newcommand{\IEqSym}{=}
\newcommand{\IEq}[2]{{#1}\IEqSym{#2}}

% Types
\newcommand{\IPower}[1]{\mathcal{P}{#1}}
\newcommand{\IPowerCl}[1]{\mathcal{P}_{\IBoxSym}{#1}}
\newcommand{\IProp}{\Omega}
\newcommand{\IPropCl}{\Omega_\IBoxSym}
\newcommand{\IHolds}[1]{|{#1}|}

%% Universe
\newcommand{\IUni}{\mathcal{U}}

%% Empty type
\newcommand{\IEmpty}{\mathbb{0}}
\newcommand{\IEmptyM}[1]{!_{#1}}

%% Unit type
\newcommand{\IUnit}{\mathbb{1}}
\newcommand{\IUnitM}[1]{{\langle\rangle}_{#1}}

%% Product types
\newcommand{\IProd}[2]{{#1}\times{#2}}
\newcommand{\IFst}{\pi_0}
\newcommand{\ISnd}{\pi_1}
\newcommand{\IPair}[2]{\langle{#1},{#2}\rangle} % pairing operator
\newcommand{\IPairBi}[2]{{#1}\times{#2}} % bifunctor action of products

%% Equaliser types
\newcommand{\IEqualiser}[2]{\mathsf{Eq({#1,#2})}}

%% Function types
\newcommand{\IFun}[2]{{{#1}\to{#2}}}
\newcommand{\ILam}[2]{{\mathop{\lambda({#1}).}{#2}}}

\newcommand{\IId}[1]{\mathbb{1}_{#1}}

% INTERNAL PCA

%% Partial application
\newcommand{\IApSym}{\cdot}
\newcommand{\IAp}[2]{{#1}\IApSym{#2}}

%% Polynomial application
\newcommand{\IPApSym}{\cdot}
\newcommand{\IPAp}[2]{{#1}\IPApSym{#2}}

%% Polynomial abstraction
\newcommand{\IAbs}[2]{\Lambda{#1}.\,{#2}}

%% Partial element ordering
\newcommand{\IPLeqSym}{\preccurlyeq}
\newcommand{\IPLeq}[2]{{#1}\IPLeqSym{#2}}

\newcommand{\IPSimSym}{\simeq}
\newcommand{\IPSim}[2]{{#1}\IPSimSym{#2}}

\newcommand{\IIdentCode}{\texttt{id}}

\newcommand{\ICompCode}{\texttt{comp}}

%% Pairing
\newcommand{\IPairCode}{\texttt{pair}}
\newcommand{\IFstCode}{\texttt{p}_0}
\newcommand{\ISndCode}{\texttt{p}_1}

%% Disjoint unions
\newcommand{\ILeftCode}{\texttt{in}_0}
\newcommand{\IRightCode}{\texttt{in}_1}
\newcommand{\IMatchCode}{\texttt{match}}


%% Sheafification
\newcommand{\ISheafSym}{\mathcal{D}}
\newcommand{\ISheaf}[1]{\ISheafSym{#1}}
\newcommand{\ISheafExt}[1]{{#1}^\sharp}
\newcommand{\ISheafIndSym}{\mathcal{D}_\text{ind}}
\newcommand{\ISheafInd}[1]{\ISheafIndSym\,{#1}}

\newcommand{\IGlueSym}{\text{glue}}
\newcommand{\IGlue}[1]{\IGlueSym_{#1}}

\newcommand{\ILeafSym}{\eta}
\newcommand{\ILeaf}[1]{\eta\,{#1}}

\newcommand{\IBranchSym}{\beta}
\newcommand{\IBranch}[3]{\beta(#1,#2,#3)}

%% Lifting monad

\newcommand{\IPar}[1]{{#1}_\bot}
\newcommand{\ITotal}[1]{{{#1}\!\downarrow}}
\newcommand{\IValSym}{\text{value}}
\newcommand{\IVal}[1]{\IValSym\,{#1}}

\newcommand{\ISquashSym}{{\Downarrow}}
\newcommand{\ISquash}[1]{\ISquashSym\,{#1}}

\newcommand{\IParExtSym}{\Diamond}
\newcommand{\IParExt}[1]{\IParExtSym{#1}}

%%%%%%%%%%%%%%%%%%%%%%%%%%%%%%%%%%%%%%%%%%%%%%%%%%%%%%%%%%%%%%%%%%%%%%%%%%%%%%%%
%% TRIPOS %%%%%%%%%%%%%%%%%%%%%%%%%%%%%%%%%%%%%%%%%%%%%%%%%%%%%%%%%%%%%%%%%%%%%%
%%%%%%%%%%%%%%%%%%%%%%%%%%%%%%%%%%%%%%%%%%%%%%%%%%%%%%%%%%%%%%%%%%%%%%%%%%%%%%%%

%% Tripos
\newcommand{\Trip}{T_\IBoxSym}

%% Realizability Predicates
\newcommand{\RealPred}[1]{\IPowerCl{(\IProd{#1}{\A})}}

% Tripos language

%% Helper macro to tag tripos language symbols with the object type
\newcommand{\TagWithObject}[2]{\overset{\scriptscriptstyle #2}{#1}}

\newcommand{\TTop}[1]{\TagWithObject{\top}{#1}}

\newcommand{\TBot}[1]{\TagWithObject{\bot}{#1}}

\newcommand{\TAndSym}[1]{\TagWithObject{\land}{#1}}
\newcommand{\TAnd}[3]{{#1}\TAndSym{#3}{#2}}

\newcommand{\TOrSym}[1]{\TagWithObject{\lor}{#1}}
\newcommand{\TOr}[3]{{#1}\TOrSym{#3}{#2}}

\newcommand{\TImpliesSym}[1]{\TagWithObject{\Rightarrow}{#1}}
\newcommand{\TImplies}[3]{{#1}\TImpliesSym{#3}{#2}}

\newcommand{\TPull}[1]{{#1}^\ast}
\newcommand{\TExists}[1]{\exists_{#1}}
\newcommand{\TForall}[1]{\forall_{#1}}

%% Category of assemblies
\newcommand{\AsmSym}{\mathbf{Asm}}
\newcommand{\Asm}[3]{\AsmSym_{(#2,#3)}(#1)}
\newcommand{\AsmCat}{\Asm{\A}{\E}{\Box}}

\newcommand{\Adj}[2]{{#1}[{#2}]}


%\makeindex

%% Front Matter
%%
% Regular title as in the article class.
%
\title{Realizability Triposes from Sheaves}

% Authors are joined by \and. Their affiliations are given by \inst, which indexes
% into the list defined using \institute
%
\author{
 Bruno da Rocha Paiva%\inst{1}}
\and
 Vincent Rahli%\inst{1}
}

% Institutes for affiliations are also joined by \and,
\institute{
  University of Birmingham, United Kingdom
 }

%  \authorrunning{} has to be set for the shorter version of the authors' names;
% otherwise a warning will be rendered in the running heads. When processed by
% EasyChair, this command is mandatory: a document without \authorrunning
% will be rejected by EasyChair

\authorrunning{da Rocha Paiva and Rahli}

% \titlerunning{} has to be set to either the main title or its shorter
% version for the running heads. When processed by
% EasyChair, this command is mandatory: a document without \titlerunning
% will be rejected by EasyChair
\titlerunning{Realizability Triposes from Sheaves}

\begin{document}

\maketitle

\begin{abstract}
  Given a topos \(\E\) and a Lawvere-Tierney topology
  \(\Box : \Omega \to \Omega\) on it, we develop a realizability \(\E\)-tripos
  using the internal logic of the topos.
  %
  Instantiating \(\E\) with a category of sheaves, we recover a notion of
  realizability with choice sequences.
\end{abstract}

% The table of contents below is added for your convenience. Please do not use
% the table of contents if you are preparing your paper for publication in the
% EPiC Series or Kalpa Publications series

%\setcounter{tocdepth}{2}
%{\small
%\tableofcontents}

%\section{To mention}
%
%Processing in EasyChair - number of pages.
%
%Examples of how EasyChair processes papers. Caveats (replacement of EC
%class, errors).

%------------------------------------------------------------------------------

%Paragraph about early investigations of Brouwer's intuitionism. Maybe
%Kreisel, Moschovakis, Kripke, etc?

It was in \textit{Brouwer's second act of intuitionism} that choice sequences
first appeared.
%
Brouwer envisioned an idealised mathematician that would
generate entries of an infinitely proceeding sequence
\((\alpha_{0}, \alpha_{1}, \alpha_{2}, \dots)\).
%
At any given moment, the mathematician would only have access to the entries
generated so far, hence any deductions would necessarily rely on a finite number
of entries.
%
The first formal systems of Brouwer's intuitionism were developed by
Kleene and Vesley~\cite{kleeneFoundationsIntuitionisticMathematics1965}
and Kreisel and Troelstra~\cite{kreiselFormalSystemsBranches1970} in which
the authors investigated the Bar Theorem, continuity principles, as well
as different kinds of choice sequences.

More recently, interpretations of Brouwer's choice sequences have been leveraged
to give anti-classical models of dependent type theories.
%
In~\cite{coquandComputationalInterpretationForcing2012} the authors give a
computational account of forcing.
%
Based on this view of forcing the authors
of~\cite{coquandIndependenceMarkovsPrinciple2017} produce a model of \MLTT{}
falsifying Markov's principle.
%
This interpretation of choice sequences is combined with term models in a series
of papers~\cite{bickfordComputabilityChurchTuringChoice2018,
  bickfordOpenBarBrouwerian2021, cohenSeparatingMarkovsPrinciples2024,
  rahliValidatingBrouwersContinuity2018, forsterMarkovsPrinciplesConstructive}
to explore principles such as bar induction, continuity of functions on the
Baire space and different versions of Markov's principles.
%
As pointed out by~\cite{sterlingHigherOrderFunctions2021}, there is a common
thread of constructions internal to sheaf models which links all the foregoing
works.
%
In the tradition of Kripke and Beth semantics, by taking a category of
sheaves over a preordered set \(\mathbb{W}\) of worlds and carrying out
the standard operational constructions internally to this model, we should
expect to recover models akin to the above.

In the sequel we will start from this observation in an attempt to connect these
realizability constructions in the form of PER models with choice sequences to
categorical realizability over categories of sheaves rather than the category of
sets.
%
As in the abstract, we fix a topos \(\E\) and a Lawvere-Tierney topology
\(\Box : \Omega \to \Omega\) and proceed to define a realizability tripos over
\(\E\).
%
We define partial combinatory algebras (pca) using \(\preccurlyeq\) as opposed
to \(\simeq\).
%
It is shown in~\cite{faberEffectiveOperationsType2016} that any
``weak'' pca, that is using \(\preccurlyeq\), is isomorphic to a ``strong'' pca,
that is using \(\simeq\), hence this is mainly an aesthetic decision.
%
For the former, we have \(a \preccurlyeq b\) if when \(a \downarrow\) then
\(b \downarrow\) and their values agree, while \(a \simeq b\) holds if
\(a \preccurlyeq b\) and \(b \preccurlyeq a\).

\begin{defn}
  An \definiendum{internal partial combinatory algebra} consists of
  an object \(\A\) of \(\E\), a partial morphism
  \(-\cdot-:\A \times \A \rightharpoonup \A\) and elements
  \(\mathsf{k},\mathsf{s} : \A\) satisfying the internal
  statements:
  \[\begin{array}{c}
      \mathsf{k} \cdot a \downarrow \qquad
      \mathsf{s} \cdot a \downarrow \qquad
      \mathsf{s} \cdot a \cdot b \downarrow \\
      a \preccurlyeq \mathsf{k} \cdot a \cdot b \qquad
      a \cdot c \cdot (b \cdot c) \preccurlyeq \mathsf{s} \cdot a \cdot b \cdot c
  \end{array}\]
\end{defn}


The usual story with partial combinatory algebras carries over to the
internal setting.
%
We can still show that a pca is functionally complete and with that we get
access to most programming constructs needed for realizability such as pairings,
booleans, coproducts, and whatever else the mind might dream of.

With an internal pca we could now define a realizability tripos akin to the
usual set-based realizability triposes.
%
In fact this is done in~\cite{vanoostenSemanticalProofJonghs1991}, in which the
author takes a pca internal to a category of presheaves, takes its
sheafification to get a pca internal to the relevant category of sheaves, and
then implicitly works in the realizability topos arising out of said pca object.
%


\begin{defn}
  Given an object \(\X\) we define the type of \definiendum{realizability
    predicates on \(\X\)} as the type \(\RealPred{\X}\).
  %
  We further define an ordering on realizability
  predicates~\(\varphi, \psi : \RealPred{\X}\) by
  %
  \[
    \varphi \leq \psi
    \IsDefined
    \IExists{e : \A}{
      \IForall{x : \X}{
        \IForall{a \in \varphi(x)}{
          \IBox{(
            \IAnd%
            {\ITotal{\IAp{e}{a}}}%
            {\IAp{e}{a} \in \psi(x)}
          )}
        }
      }
    }
  \]
  %
  %We will say that \definiendum{\(e\) evidences \(\varphi \leq \psi\)} if
  %it witnesses the existential quantifier above.
\end{defn}

Using the internal language of \(\E\) we can show that realizability predicates
on an object \(\X\) with this particular ordering form a pre-ordered set.
%
Furthermore, we can define reindexing pre-Heyting algebra morphisms by
precomposition, show these have left and right adjoints (as monotone maps)
satisfying the Beck-Chevalley condition, and finally we can give an appropriate
generic element giving us an \(\E\)-tripos.

Instantiating \(\E\) to the category of presheaves over a poset \(\mathbb{W}\)
and using the \(\Box\) arising from some notion of sheaves, we recover a notion
of realizability with choice sequences similar to the notions studied
previously.
%
The category of assemblies arising out of this tripos seems like a particularly
good setting for studying the realizability of choice sequences themselves.
%
By choosing different pca objects we expect to be able to handle different
versions of choice sequences, from fully lawless choice sequences, to lawful
choice sequences and any variation sitting in between these.
%
This contrasts the situation with a sheafified pca object, as by sheafifying
we inadvertently add realizers for many choice sequences which may not
have been realized before.
%
Towards this goal, we now suggest two assemblies of interest: that of
pure natural numbers and that of effectful natural numbers.

\begin{defn}
  The \definiendum{pure natural numbers assembly}, denoted
  \(\mathsf{N}_\mathsf{pur}\), consists of the constant presheaf
  \(\Delta\mathbb{N}\) along with the realizability relation \(e \vDash_{w} n\)
  if and only if the code \(e : \A_{w}\) equals the Church encoding of
  \(n : \mathbb{N}\) at \(w\).
\end{defn}

\begin{defn}
  The \definiendum{effectful natural numbers assembly}, denoted
  \(\mathsf{N}_\mathsf{eff}\), consists of the sheafification of \(\Delta\mathbb{N}\).
  %
  As for the realizability relation, if we have \(\mathcal{U}\) covering \(w\)
  and \(n : \mathcal{U} \to \mathbb{N}\), then
  \(e \vDash_{w} (\mathcal{U}, n)\), if and only if there exists a cover
  \(\mathcal{V}\) of \(w\) such that for all
  \(v \in \mathcal{U} \cap \mathcal{V}\) and \(a : \A_{v}\),
  \({e|_{v}} \cdot_{v} a\) is defined and equals the Church encoding of
  \(n_{v}\) at \(v\).
\end{defn}

At the level of underlying presheaves, the later is the sheafification of the
former, so we hope to find an analogous universal property to justify it as the
correct definition of effectful natural numbers in this setting.
%
With this, we can then study choice sequences as the exponential object
\(\IFun{\mathsf{N}_{\mathsf{eff}}}{\mathsf{N}_{\mathsf{eff}}}\) in this category.

\bibliographystyle{plain}
%\bibliographystyle{alpha}
%\bibliographystyle{unsrt}
%\bibliographystyle{abbrv}
\bibliography{references}

%------------------------------------------------------------------------------
\end{document}

